
% Default to the notebook output style

    


% Inherit from the specified cell style.




    
\documentclass[11pt]{article}

    
    
    \usepackage[T1]{fontenc}
    % Nicer default font (+ math font) than Computer Modern for most use cases
    \usepackage{mathpazo}

    % Basic figure setup, for now with no caption control since it's done
    % automatically by Pandoc (which extracts ![](path) syntax from Markdown).
    \usepackage{graphicx}
    % We will generate all images so they have a width \maxwidth. This means
    % that they will get their normal width if they fit onto the page, but
    % are scaled down if they would overflow the margins.
    \makeatletter
    \def\maxwidth{\ifdim\Gin@nat@width>\linewidth\linewidth
    \else\Gin@nat@width\fi}
    \makeatother
    \let\Oldincludegraphics\includegraphics
    % Set max figure width to be 80% of text width, for now hardcoded.
    \renewcommand{\includegraphics}[1]{\Oldincludegraphics[width=.8\maxwidth]{#1}}
    % Ensure that by default, figures have no caption (until we provide a
    % proper Figure object with a Caption API and a way to capture that
    % in the conversion process - todo).
    \usepackage{caption}
    \DeclareCaptionLabelFormat{nolabel}{}
    \captionsetup{labelformat=nolabel}

    \usepackage{adjustbox} % Used to constrain images to a maximum size 
    \usepackage{xcolor} % Allow colors to be defined
    \usepackage{enumerate} % Needed for markdown enumerations to work
    \usepackage{geometry} % Used to adjust the document margins
    \usepackage{amsmath} % Equations
    \usepackage{amssymb} % Equations
    \usepackage{textcomp} % defines textquotesingle
    % Hack from http://tex.stackexchange.com/a/47451/13684:
    \AtBeginDocument{%
        \def\PYZsq{\textquotesingle}% Upright quotes in Pygmentized code
    }
    \usepackage{upquote} % Upright quotes for verbatim code
    \usepackage{eurosym} % defines \euro
    \usepackage[mathletters]{ucs} % Extended unicode (utf-8) support
    \usepackage[utf8x]{inputenc} % Allow utf-8 characters in the tex document
    \usepackage{fancyvrb} % verbatim replacement that allows latex
    \usepackage{grffile} % extends the file name processing of package graphics 
                         % to support a larger range 
    % The hyperref package gives us a pdf with properly built
    % internal navigation ('pdf bookmarks' for the table of contents,
    % internal cross-reference links, web links for URLs, etc.)
    \usepackage{hyperref}
    \usepackage{longtable} % longtable support required by pandoc >1.10
    \usepackage{booktabs}  % table support for pandoc > 1.12.2
    \usepackage[inline]{enumitem} % IRkernel/repr support (it uses the enumerate* environment)
    \usepackage[normalem]{ulem} % ulem is needed to support strikethroughs (\sout)
                                % normalem makes italics be italics, not underlines
    

    
    
    % Colors for the hyperref package
    \definecolor{urlcolor}{rgb}{0,.145,.698}
    \definecolor{linkcolor}{rgb}{.71,0.21,0.01}
    \definecolor{citecolor}{rgb}{.12,.54,.11}

    % ANSI colors
    \definecolor{ansi-black}{HTML}{3E424D}
    \definecolor{ansi-black-intense}{HTML}{282C36}
    \definecolor{ansi-red}{HTML}{E75C58}
    \definecolor{ansi-red-intense}{HTML}{B22B31}
    \definecolor{ansi-green}{HTML}{00A250}
    \definecolor{ansi-green-intense}{HTML}{007427}
    \definecolor{ansi-yellow}{HTML}{DDB62B}
    \definecolor{ansi-yellow-intense}{HTML}{B27D12}
    \definecolor{ansi-blue}{HTML}{208FFB}
    \definecolor{ansi-blue-intense}{HTML}{0065CA}
    \definecolor{ansi-magenta}{HTML}{D160C4}
    \definecolor{ansi-magenta-intense}{HTML}{A03196}
    \definecolor{ansi-cyan}{HTML}{60C6C8}
    \definecolor{ansi-cyan-intense}{HTML}{258F8F}
    \definecolor{ansi-white}{HTML}{C5C1B4}
    \definecolor{ansi-white-intense}{HTML}{A1A6B2}

    % commands and environments needed by pandoc snippets
    % extracted from the output of `pandoc -s`
    \providecommand{\tightlist}{%
      \setlength{\itemsep}{0pt}\setlength{\parskip}{0pt}}
    \DefineVerbatimEnvironment{Highlighting}{Verbatim}{commandchars=\\\{\}}
    % Add ',fontsize=\small' for more characters per line
    \newenvironment{Shaded}{}{}
    \newcommand{\KeywordTok}[1]{\textcolor[rgb]{0.00,0.44,0.13}{\textbf{{#1}}}}
    \newcommand{\DataTypeTok}[1]{\textcolor[rgb]{0.56,0.13,0.00}{{#1}}}
    \newcommand{\DecValTok}[1]{\textcolor[rgb]{0.25,0.63,0.44}{{#1}}}
    \newcommand{\BaseNTok}[1]{\textcolor[rgb]{0.25,0.63,0.44}{{#1}}}
    \newcommand{\FloatTok}[1]{\textcolor[rgb]{0.25,0.63,0.44}{{#1}}}
    \newcommand{\CharTok}[1]{\textcolor[rgb]{0.25,0.44,0.63}{{#1}}}
    \newcommand{\StringTok}[1]{\textcolor[rgb]{0.25,0.44,0.63}{{#1}}}
    \newcommand{\CommentTok}[1]{\textcolor[rgb]{0.38,0.63,0.69}{\textit{{#1}}}}
    \newcommand{\OtherTok}[1]{\textcolor[rgb]{0.00,0.44,0.13}{{#1}}}
    \newcommand{\AlertTok}[1]{\textcolor[rgb]{1.00,0.00,0.00}{\textbf{{#1}}}}
    \newcommand{\FunctionTok}[1]{\textcolor[rgb]{0.02,0.16,0.49}{{#1}}}
    \newcommand{\RegionMarkerTok}[1]{{#1}}
    \newcommand{\ErrorTok}[1]{\textcolor[rgb]{1.00,0.00,0.00}{\textbf{{#1}}}}
    \newcommand{\NormalTok}[1]{{#1}}
    
    % Additional commands for more recent versions of Pandoc
    \newcommand{\ConstantTok}[1]{\textcolor[rgb]{0.53,0.00,0.00}{{#1}}}
    \newcommand{\SpecialCharTok}[1]{\textcolor[rgb]{0.25,0.44,0.63}{{#1}}}
    \newcommand{\VerbatimStringTok}[1]{\textcolor[rgb]{0.25,0.44,0.63}{{#1}}}
    \newcommand{\SpecialStringTok}[1]{\textcolor[rgb]{0.73,0.40,0.53}{{#1}}}
    \newcommand{\ImportTok}[1]{{#1}}
    \newcommand{\DocumentationTok}[1]{\textcolor[rgb]{0.73,0.13,0.13}{\textit{{#1}}}}
    \newcommand{\AnnotationTok}[1]{\textcolor[rgb]{0.38,0.63,0.69}{\textbf{\textit{{#1}}}}}
    \newcommand{\CommentVarTok}[1]{\textcolor[rgb]{0.38,0.63,0.69}{\textbf{\textit{{#1}}}}}
    \newcommand{\VariableTok}[1]{\textcolor[rgb]{0.10,0.09,0.49}{{#1}}}
    \newcommand{\ControlFlowTok}[1]{\textcolor[rgb]{0.00,0.44,0.13}{\textbf{{#1}}}}
    \newcommand{\OperatorTok}[1]{\textcolor[rgb]{0.40,0.40,0.40}{{#1}}}
    \newcommand{\BuiltInTok}[1]{{#1}}
    \newcommand{\ExtensionTok}[1]{{#1}}
    \newcommand{\PreprocessorTok}[1]{\textcolor[rgb]{0.74,0.48,0.00}{{#1}}}
    \newcommand{\AttributeTok}[1]{\textcolor[rgb]{0.49,0.56,0.16}{{#1}}}
    \newcommand{\InformationTok}[1]{\textcolor[rgb]{0.38,0.63,0.69}{\textbf{\textit{{#1}}}}}
    \newcommand{\WarningTok}[1]{\textcolor[rgb]{0.38,0.63,0.69}{\textbf{\textit{{#1}}}}}
    
    
    % Define a nice break command that doesn't care if a line doesn't already
    % exist.
    \def\br{\hspace*{\fill} \\* }
    % Math Jax compatability definitions
    \def\gt{>}
    \def\lt{<}
    % Document parameters
    \title{AdvancedLane}
    
    
    

    % Pygments definitions
    
\makeatletter
\def\PY@reset{\let\PY@it=\relax \let\PY@bf=\relax%
    \let\PY@ul=\relax \let\PY@tc=\relax%
    \let\PY@bc=\relax \let\PY@ff=\relax}
\def\PY@tok#1{\csname PY@tok@#1\endcsname}
\def\PY@toks#1+{\ifx\relax#1\empty\else%
    \PY@tok{#1}\expandafter\PY@toks\fi}
\def\PY@do#1{\PY@bc{\PY@tc{\PY@ul{%
    \PY@it{\PY@bf{\PY@ff{#1}}}}}}}
\def\PY#1#2{\PY@reset\PY@toks#1+\relax+\PY@do{#2}}

\expandafter\def\csname PY@tok@se\endcsname{\let\PY@bf=\textbf\def\PY@tc##1{\textcolor[rgb]{0.73,0.40,0.13}{##1}}}
\expandafter\def\csname PY@tok@sc\endcsname{\def\PY@tc##1{\textcolor[rgb]{0.73,0.13,0.13}{##1}}}
\expandafter\def\csname PY@tok@cs\endcsname{\let\PY@it=\textit\def\PY@tc##1{\textcolor[rgb]{0.25,0.50,0.50}{##1}}}
\expandafter\def\csname PY@tok@nl\endcsname{\def\PY@tc##1{\textcolor[rgb]{0.63,0.63,0.00}{##1}}}
\expandafter\def\csname PY@tok@mo\endcsname{\def\PY@tc##1{\textcolor[rgb]{0.40,0.40,0.40}{##1}}}
\expandafter\def\csname PY@tok@no\endcsname{\def\PY@tc##1{\textcolor[rgb]{0.53,0.00,0.00}{##1}}}
\expandafter\def\csname PY@tok@na\endcsname{\def\PY@tc##1{\textcolor[rgb]{0.49,0.56,0.16}{##1}}}
\expandafter\def\csname PY@tok@s2\endcsname{\def\PY@tc##1{\textcolor[rgb]{0.73,0.13,0.13}{##1}}}
\expandafter\def\csname PY@tok@ow\endcsname{\let\PY@bf=\textbf\def\PY@tc##1{\textcolor[rgb]{0.67,0.13,1.00}{##1}}}
\expandafter\def\csname PY@tok@bp\endcsname{\def\PY@tc##1{\textcolor[rgb]{0.00,0.50,0.00}{##1}}}
\expandafter\def\csname PY@tok@kt\endcsname{\def\PY@tc##1{\textcolor[rgb]{0.69,0.00,0.25}{##1}}}
\expandafter\def\csname PY@tok@sh\endcsname{\def\PY@tc##1{\textcolor[rgb]{0.73,0.13,0.13}{##1}}}
\expandafter\def\csname PY@tok@gi\endcsname{\def\PY@tc##1{\textcolor[rgb]{0.00,0.63,0.00}{##1}}}
\expandafter\def\csname PY@tok@err\endcsname{\def\PY@bc##1{\setlength{\fboxsep}{0pt}\fcolorbox[rgb]{1.00,0.00,0.00}{1,1,1}{\strut ##1}}}
\expandafter\def\csname PY@tok@s\endcsname{\def\PY@tc##1{\textcolor[rgb]{0.73,0.13,0.13}{##1}}}
\expandafter\def\csname PY@tok@si\endcsname{\let\PY@bf=\textbf\def\PY@tc##1{\textcolor[rgb]{0.73,0.40,0.53}{##1}}}
\expandafter\def\csname PY@tok@sd\endcsname{\let\PY@it=\textit\def\PY@tc##1{\textcolor[rgb]{0.73,0.13,0.13}{##1}}}
\expandafter\def\csname PY@tok@nb\endcsname{\def\PY@tc##1{\textcolor[rgb]{0.00,0.50,0.00}{##1}}}
\expandafter\def\csname PY@tok@kr\endcsname{\let\PY@bf=\textbf\def\PY@tc##1{\textcolor[rgb]{0.00,0.50,0.00}{##1}}}
\expandafter\def\csname PY@tok@ge\endcsname{\let\PY@it=\textit}
\expandafter\def\csname PY@tok@ne\endcsname{\let\PY@bf=\textbf\def\PY@tc##1{\textcolor[rgb]{0.82,0.25,0.23}{##1}}}
\expandafter\def\csname PY@tok@cp\endcsname{\def\PY@tc##1{\textcolor[rgb]{0.74,0.48,0.00}{##1}}}
\expandafter\def\csname PY@tok@il\endcsname{\def\PY@tc##1{\textcolor[rgb]{0.40,0.40,0.40}{##1}}}
\expandafter\def\csname PY@tok@mb\endcsname{\def\PY@tc##1{\textcolor[rgb]{0.40,0.40,0.40}{##1}}}
\expandafter\def\csname PY@tok@gt\endcsname{\def\PY@tc##1{\textcolor[rgb]{0.00,0.27,0.87}{##1}}}
\expandafter\def\csname PY@tok@mf\endcsname{\def\PY@tc##1{\textcolor[rgb]{0.40,0.40,0.40}{##1}}}
\expandafter\def\csname PY@tok@kp\endcsname{\def\PY@tc##1{\textcolor[rgb]{0.00,0.50,0.00}{##1}}}
\expandafter\def\csname PY@tok@nt\endcsname{\let\PY@bf=\textbf\def\PY@tc##1{\textcolor[rgb]{0.00,0.50,0.00}{##1}}}
\expandafter\def\csname PY@tok@mh\endcsname{\def\PY@tc##1{\textcolor[rgb]{0.40,0.40,0.40}{##1}}}
\expandafter\def\csname PY@tok@gr\endcsname{\def\PY@tc##1{\textcolor[rgb]{1.00,0.00,0.00}{##1}}}
\expandafter\def\csname PY@tok@nn\endcsname{\let\PY@bf=\textbf\def\PY@tc##1{\textcolor[rgb]{0.00,0.00,1.00}{##1}}}
\expandafter\def\csname PY@tok@m\endcsname{\def\PY@tc##1{\textcolor[rgb]{0.40,0.40,0.40}{##1}}}
\expandafter\def\csname PY@tok@vi\endcsname{\def\PY@tc##1{\textcolor[rgb]{0.10,0.09,0.49}{##1}}}
\expandafter\def\csname PY@tok@sb\endcsname{\def\PY@tc##1{\textcolor[rgb]{0.73,0.13,0.13}{##1}}}
\expandafter\def\csname PY@tok@fm\endcsname{\def\PY@tc##1{\textcolor[rgb]{0.00,0.00,1.00}{##1}}}
\expandafter\def\csname PY@tok@c1\endcsname{\let\PY@it=\textit\def\PY@tc##1{\textcolor[rgb]{0.25,0.50,0.50}{##1}}}
\expandafter\def\csname PY@tok@cm\endcsname{\let\PY@it=\textit\def\PY@tc##1{\textcolor[rgb]{0.25,0.50,0.50}{##1}}}
\expandafter\def\csname PY@tok@nc\endcsname{\let\PY@bf=\textbf\def\PY@tc##1{\textcolor[rgb]{0.00,0.00,1.00}{##1}}}
\expandafter\def\csname PY@tok@cpf\endcsname{\let\PY@it=\textit\def\PY@tc##1{\textcolor[rgb]{0.25,0.50,0.50}{##1}}}
\expandafter\def\csname PY@tok@mi\endcsname{\def\PY@tc##1{\textcolor[rgb]{0.40,0.40,0.40}{##1}}}
\expandafter\def\csname PY@tok@kd\endcsname{\let\PY@bf=\textbf\def\PY@tc##1{\textcolor[rgb]{0.00,0.50,0.00}{##1}}}
\expandafter\def\csname PY@tok@gh\endcsname{\let\PY@bf=\textbf\def\PY@tc##1{\textcolor[rgb]{0.00,0.00,0.50}{##1}}}
\expandafter\def\csname PY@tok@nf\endcsname{\def\PY@tc##1{\textcolor[rgb]{0.00,0.00,1.00}{##1}}}
\expandafter\def\csname PY@tok@dl\endcsname{\def\PY@tc##1{\textcolor[rgb]{0.73,0.13,0.13}{##1}}}
\expandafter\def\csname PY@tok@o\endcsname{\def\PY@tc##1{\textcolor[rgb]{0.40,0.40,0.40}{##1}}}
\expandafter\def\csname PY@tok@w\endcsname{\def\PY@tc##1{\textcolor[rgb]{0.73,0.73,0.73}{##1}}}
\expandafter\def\csname PY@tok@ch\endcsname{\let\PY@it=\textit\def\PY@tc##1{\textcolor[rgb]{0.25,0.50,0.50}{##1}}}
\expandafter\def\csname PY@tok@kn\endcsname{\let\PY@bf=\textbf\def\PY@tc##1{\textcolor[rgb]{0.00,0.50,0.00}{##1}}}
\expandafter\def\csname PY@tok@nd\endcsname{\def\PY@tc##1{\textcolor[rgb]{0.67,0.13,1.00}{##1}}}
\expandafter\def\csname PY@tok@sa\endcsname{\def\PY@tc##1{\textcolor[rgb]{0.73,0.13,0.13}{##1}}}
\expandafter\def\csname PY@tok@vm\endcsname{\def\PY@tc##1{\textcolor[rgb]{0.10,0.09,0.49}{##1}}}
\expandafter\def\csname PY@tok@c\endcsname{\let\PY@it=\textit\def\PY@tc##1{\textcolor[rgb]{0.25,0.50,0.50}{##1}}}
\expandafter\def\csname PY@tok@gd\endcsname{\def\PY@tc##1{\textcolor[rgb]{0.63,0.00,0.00}{##1}}}
\expandafter\def\csname PY@tok@vg\endcsname{\def\PY@tc##1{\textcolor[rgb]{0.10,0.09,0.49}{##1}}}
\expandafter\def\csname PY@tok@gs\endcsname{\let\PY@bf=\textbf}
\expandafter\def\csname PY@tok@gp\endcsname{\let\PY@bf=\textbf\def\PY@tc##1{\textcolor[rgb]{0.00,0.00,0.50}{##1}}}
\expandafter\def\csname PY@tok@vc\endcsname{\def\PY@tc##1{\textcolor[rgb]{0.10,0.09,0.49}{##1}}}
\expandafter\def\csname PY@tok@sr\endcsname{\def\PY@tc##1{\textcolor[rgb]{0.73,0.40,0.53}{##1}}}
\expandafter\def\csname PY@tok@nv\endcsname{\def\PY@tc##1{\textcolor[rgb]{0.10,0.09,0.49}{##1}}}
\expandafter\def\csname PY@tok@kc\endcsname{\let\PY@bf=\textbf\def\PY@tc##1{\textcolor[rgb]{0.00,0.50,0.00}{##1}}}
\expandafter\def\csname PY@tok@k\endcsname{\let\PY@bf=\textbf\def\PY@tc##1{\textcolor[rgb]{0.00,0.50,0.00}{##1}}}
\expandafter\def\csname PY@tok@ni\endcsname{\let\PY@bf=\textbf\def\PY@tc##1{\textcolor[rgb]{0.60,0.60,0.60}{##1}}}
\expandafter\def\csname PY@tok@ss\endcsname{\def\PY@tc##1{\textcolor[rgb]{0.10,0.09,0.49}{##1}}}
\expandafter\def\csname PY@tok@s1\endcsname{\def\PY@tc##1{\textcolor[rgb]{0.73,0.13,0.13}{##1}}}
\expandafter\def\csname PY@tok@gu\endcsname{\let\PY@bf=\textbf\def\PY@tc##1{\textcolor[rgb]{0.50,0.00,0.50}{##1}}}
\expandafter\def\csname PY@tok@sx\endcsname{\def\PY@tc##1{\textcolor[rgb]{0.00,0.50,0.00}{##1}}}
\expandafter\def\csname PY@tok@go\endcsname{\def\PY@tc##1{\textcolor[rgb]{0.53,0.53,0.53}{##1}}}

\def\PYZbs{\char`\\}
\def\PYZus{\char`\_}
\def\PYZob{\char`\{}
\def\PYZcb{\char`\}}
\def\PYZca{\char`\^}
\def\PYZam{\char`\&}
\def\PYZlt{\char`\<}
\def\PYZgt{\char`\>}
\def\PYZsh{\char`\#}
\def\PYZpc{\char`\%}
\def\PYZdl{\char`\$}
\def\PYZhy{\char`\-}
\def\PYZsq{\char`\'}
\def\PYZdq{\char`\"}
\def\PYZti{\char`\~}
% for compatibility with earlier versions
\def\PYZat{@}
\def\PYZlb{[}
\def\PYZrb{]}
\makeatother


    % Exact colors from NB
    \definecolor{incolor}{rgb}{0.0, 0.0, 0.5}
    \definecolor{outcolor}{rgb}{0.545, 0.0, 0.0}



    
    % Prevent overflowing lines due to hard-to-break entities
    \sloppy 
    % Setup hyperref package
    \hypersetup{
      breaklinks=true,  % so long urls are correctly broken across lines
      colorlinks=true,
      urlcolor=urlcolor,
      linkcolor=linkcolor,
      citecolor=citecolor,
      }
    % Slightly bigger margins than the latex defaults
    
    \geometry{verbose,tmargin=1in,bmargin=1in,lmargin=1in,rmargin=1in}
    
    

    \begin{document}
    
    
    \maketitle
    
    

    
    \hypertarget{advanced-lane-finding-project}{%
\subsection{Advanced Lane Finding
Project}\label{advanced-lane-finding-project}}

The goals / steps of this project are the following:

\begin{itemize}
\tightlist
\item
  Compute the camera calibration matrix and distortion coefficients
  given a set of chessboard images.
\item
  Apply a distortion correction to raw images.
\item
  Use color transforms, gradients, etc., to create a thresholded binary
  image.
\item
  Apply a perspective transform to rectify binary image (``birds-eye
  view'').
\item
  Detect lane pixels and fit to find the lane boundary.
\item
  Determine the curvature of the lane and vehicle position with respect
  to center.
\item
  Warp the detected lane boundaries back onto the original image.
\item
  Output visual display of the lane boundaries and numerical estimation
  of lane curvature and vehicle position.
\end{itemize}

\begin{center}\rule{0.5\linewidth}{\linethickness}\end{center}

\hypertarget{camera-calibration-using-the-chessboard-images}{%
\subsection{Camera calibration using the Chessboard
images}\label{camera-calibration-using-the-chessboard-images}}

    \begin{Verbatim}[commandchars=\\\{\}]
{\color{incolor}In [{\color{incolor}1}]:} \PY{c+c1}{\PYZsh{} Importing libraries}
        \PY{k+kn}{import} \PY{n+nn}{numpy} \PY{k}{as} \PY{n+nn}{np}
        \PY{k+kn}{import} \PY{n+nn}{cv2}
        \PY{k+kn}{import} \PY{n+nn}{os}
        \PY{k+kn}{import} \PY{n+nn}{glob}
        \PY{k+kn}{import} \PY{n+nn}{matplotlib}\PY{n+nn}{.}\PY{n+nn}{pyplot} \PY{k}{as} \PY{n+nn}{plt}
        \PY{k+kn}{import} \PY{n+nn}{matplotlib}\PY{n+nn}{.}\PY{n+nn}{image} \PY{k}{as} \PY{n+nn}{mpimg}
        \PY{c+c1}{\PYZsh{}\PYZpc{}matplotlib qt}
\end{Verbatim}


    \begin{Verbatim}[commandchars=\\\{\}]
{\color{incolor}In [{\color{incolor}12}]:} \PY{c+c1}{\PYZsh{} 1. CALIBRATING camera}
         \PY{c+c1}{\PYZsh{} prepare object points, like (0,0,0), (1,0,0), (2,0,0) ....,(6,5,0)}
         \PY{n}{objp} \PY{o}{=} \PY{n}{np}\PY{o}{.}\PY{n}{zeros}\PY{p}{(}\PY{p}{(}\PY{l+m+mi}{6}\PY{o}{*}\PY{l+m+mi}{9}\PY{p}{,}\PY{l+m+mi}{3}\PY{p}{)}\PY{p}{,} \PY{n}{np}\PY{o}{.}\PY{n}{float32}\PY{p}{)}
         \PY{n}{objp}\PY{p}{[}\PY{p}{:}\PY{p}{,}\PY{p}{:}\PY{l+m+mi}{2}\PY{p}{]} \PY{o}{=} \PY{n}{np}\PY{o}{.}\PY{n}{mgrid}\PY{p}{[}\PY{l+m+mi}{0}\PY{p}{:}\PY{l+m+mi}{9}\PY{p}{,}\PY{l+m+mi}{0}\PY{p}{:}\PY{l+m+mi}{6}\PY{p}{]}\PY{o}{.}\PY{n}{T}\PY{o}{.}\PY{n}{reshape}\PY{p}{(}\PY{o}{\PYZhy{}}\PY{l+m+mi}{1}\PY{p}{,}\PY{l+m+mi}{2}\PY{p}{)}
         
         \PY{c+c1}{\PYZsh{} Arrays to store object points and image points from all the images.}
         \PY{n}{objpoints} \PY{o}{=} \PY{p}{[}\PY{p}{]} \PY{c+c1}{\PYZsh{} 3d points in real world space}
         \PY{n}{imgpoints} \PY{o}{=} \PY{p}{[}\PY{p}{]} \PY{c+c1}{\PYZsh{} 2d points in image plane.}
         
         \PY{c+c1}{\PYZsh{} Make a list of calibration images}
         \PY{n}{images} \PY{o}{=} \PY{n}{glob}\PY{o}{.}\PY{n}{glob}\PY{p}{(}\PY{n}{os}\PY{o}{.}\PY{n}{path}\PY{o}{.}\PY{n}{join}\PY{p}{(}\PY{l+s+s1}{\PYZsq{}}\PY{l+s+s1}{camera\PYZus{}cal}\PY{l+s+s1}{\PYZsq{}}\PY{p}{,} \PY{l+s+s1}{\PYZsq{}}\PY{l+s+s1}{calibration*.jpg}\PY{l+s+s1}{\PYZsq{}}\PY{p}{)}\PY{p}{)}
         
         \PY{c+c1}{\PYZsh{} Step through the list and search for chessboard corners}
         \PY{k}{for} \PY{n}{fname} \PY{o+ow}{in} \PY{n}{images}\PY{p}{:}
             \PY{n}{img} \PY{o}{=} \PY{n}{cv2}\PY{o}{.}\PY{n}{imread}\PY{p}{(}\PY{n}{fname}\PY{p}{)}
             \PY{n}{gray} \PY{o}{=} \PY{n}{cv2}\PY{o}{.}\PY{n}{cvtColor}\PY{p}{(}\PY{n}{img}\PY{p}{,}\PY{n}{cv2}\PY{o}{.}\PY{n}{COLOR\PYZus{}BGR2GRAY}\PY{p}{)}
             \PY{c+c1}{\PYZsh{}print(gray)}
             \PY{c+c1}{\PYZsh{}plt.imshow(img)}
             \PY{c+c1}{\PYZsh{}plt.show()}
             \PY{c+c1}{\PYZsh{} Find the chessboard corners}
             \PY{n}{ret}\PY{p}{,} \PY{n}{corners} \PY{o}{=} \PY{n}{cv2}\PY{o}{.}\PY{n}{findChessboardCorners}\PY{p}{(}\PY{n}{gray}\PY{p}{,} \PY{p}{(}\PY{l+m+mi}{9}\PY{p}{,}\PY{l+m+mi}{6}\PY{p}{)}\PY{p}{,}\PY{k+kc}{None}\PY{p}{)}
         
             \PY{c+c1}{\PYZsh{} If found, add object points, image points}
             \PY{k}{if} \PY{n}{ret} \PY{o}{==} \PY{k+kc}{True}\PY{p}{:}
                 \PY{n}{objpoints}\PY{o}{.}\PY{n}{append}\PY{p}{(}\PY{n}{objp}\PY{p}{)}
                 \PY{n}{imgpoints}\PY{o}{.}\PY{n}{append}\PY{p}{(}\PY{n}{corners}\PY{p}{)}
         
                 \PY{c+c1}{\PYZsh{} Draw and display the corners}
                 \PY{n}{img} \PY{o}{=} \PY{n}{cv2}\PY{o}{.}\PY{n}{drawChessboardCorners}\PY{p}{(}\PY{n}{img}\PY{p}{,} \PY{p}{(}\PY{l+m+mi}{9}\PY{p}{,}\PY{l+m+mi}{6}\PY{p}{)}\PY{p}{,} \PY{n}{corners}\PY{p}{,} \PY{n}{ret}\PY{p}{)}
                 \PY{c+c1}{\PYZsh{}cv2.imshow(\PYZsq{}img\PYZsq{},img)}
                 \PY{c+c1}{\PYZsh{}cv2.waitKey(500)}
         
         \PY{c+c1}{\PYZsh{} Calibrating camera and collecting distortion parameters}
         \PY{n}{ret}\PY{p}{,} \PY{n}{mtx}\PY{p}{,} \PY{n}{dist}\PY{p}{,} \PY{n}{rvecs}\PY{p}{,} \PY{n}{tvecs} \PY{o}{=} \PY{n}{cv2}\PY{o}{.}\PY{n}{calibrateCamera}\PY{p}{(}\PY{n}{objpoints}\PY{p}{,} \PY{n}{imgpoints}\PY{p}{,} \PY{n}{gray}\PY{o}{.}\PY{n}{shape}\PY{p}{[}\PY{p}{:}\PY{p}{:}\PY{o}{\PYZhy{}}\PY{l+m+mi}{1}\PY{p}{]}\PY{p}{,} \PY{k+kc}{None}\PY{p}{,} \PY{k+kc}{None}\PY{p}{)}
         \PY{n}{cv2}\PY{o}{.}\PY{n}{destroyAllWindows}\PY{p}{(}\PY{p}{)}
\end{Verbatim}


    \begin{Verbatim}[commandchars=\\\{\}]
{\color{incolor}In [{\color{incolor}28}]:} \PY{c+c1}{\PYZsh{} 2. UNDISTORTING test image}
         \PY{n}{img} \PY{o}{=} \PY{n}{mpimg}\PY{o}{.}\PY{n}{imread}\PY{p}{(}\PY{l+s+s1}{\PYZsq{}}\PY{l+s+s1}{test\PYZus{}images/test1.jpg}\PY{l+s+s1}{\PYZsq{}}\PY{p}{)}
         \PY{n}{plt}\PY{o}{.}\PY{n}{imshow}\PY{p}{(}\PY{n}{img}\PY{p}{)}
         \PY{n}{dst} \PY{o}{=} \PY{n}{cv2}\PY{o}{.}\PY{n}{undistort}\PY{p}{(}\PY{n}{img}\PY{p}{,} \PY{n}{mtx}\PY{p}{,} \PY{n}{dist}\PY{p}{,} \PY{k+kc}{None}\PY{p}{,} \PY{n}{mtx}\PY{p}{)}
         \PY{c+c1}{\PYZsh{} Converting the image to RGB and saving the undistorted image}
         \PY{n}{dst\PYZus{}rgb} \PY{o}{=} \PY{n}{cv2}\PY{o}{.}\PY{n}{cvtColor}\PY{p}{(}\PY{n}{dst}\PY{p}{,}\PY{n}{cv2}\PY{o}{.}\PY{n}{COLOR\PYZus{}BGR2RGB}\PY{p}{)}
         \PY{n}{cv2}\PY{o}{.}\PY{n}{imwrite}\PY{p}{(}\PY{l+s+s1}{\PYZsq{}}\PY{l+s+s1}{output\PYZus{}images/test\PYZus{}dst.jpg}\PY{l+s+s1}{\PYZsq{}}\PY{p}{,}\PY{n}{dst\PYZus{}rgb}\PY{p}{)}
         \PY{c+c1}{\PYZsh{}cv2.imwrite(\PYZsq{}output\PYZus{}images/test\PYZus{}img.jpg\PYZsq{},img)}
\end{Verbatim}


\begin{Verbatim}[commandchars=\\\{\}]
{\color{outcolor}Out[{\color{outcolor}28}]:} True
\end{Verbatim}
            
    \begin{center}
    \adjustimage{max size={0.9\linewidth}{0.9\paperheight}}{output_3_1.png}
    \end{center}
    { \hspace*{\fill} \\}
    
    \begin{Verbatim}[commandchars=\\\{\}]
{\color{incolor}In [{\color{incolor} }]:} \PY{c+c1}{\PYZsh{} 3. GRADIENT and COLOR transform}
        \PY{c+c1}{\PYZsh{} Re\PYZhy{}using the functions created for lesson 6. Combining Thresholds}
        \PY{c+c1}{\PYZsh{} Three Sobel transforms combined for the most reliable outline identification}
\end{Verbatim}


    \begin{Verbatim}[commandchars=\\\{\}]
{\color{incolor}In [{\color{incolor}30}]:} \PY{c+c1}{\PYZsh{} Reading back the undistorted test image for testing Sobel transform functions}
         \PY{n}{image} \PY{o}{=} \PY{n}{mpimg}\PY{o}{.}\PY{n}{imread}\PY{p}{(}\PY{l+s+s1}{\PYZsq{}}\PY{l+s+s1}{output\PYZus{}images/test\PYZus{}dst.jpg}\PY{l+s+s1}{\PYZsq{}}\PY{p}{)}
         
         \PY{c+c1}{\PYZsh{}printing out some stats and plotting}
         \PY{n+nb}{print}\PY{p}{(}\PY{l+s+s1}{\PYZsq{}}\PY{l+s+s1}{This image is:}\PY{l+s+s1}{\PYZsq{}}\PY{p}{,} \PY{n+nb}{type}\PY{p}{(}\PY{n}{image}\PY{p}{)}\PY{p}{,} \PY{l+s+s1}{\PYZsq{}}\PY{l+s+s1}{with dimensions:}\PY{l+s+s1}{\PYZsq{}}\PY{p}{,} \PY{n}{image}\PY{o}{.}\PY{n}{shape}\PY{p}{)}
         \PY{n}{plt}\PY{o}{.}\PY{n}{imshow}\PY{p}{(}\PY{n}{image}\PY{p}{)}  \PY{c+c1}{\PYZsh{} if you wanted to show a single color channel image called \PYZsq{}gray\PYZsq{}, for example, call as plt.imshow(gray, cmap=\PYZsq{}gray\PYZsq{})}
\end{Verbatim}


    \begin{Verbatim}[commandchars=\\\{\}]
This image is: <class 'numpy.ndarray'> with dimensions: (720, 1280, 3)

    \end{Verbatim}

\begin{Verbatim}[commandchars=\\\{\}]
{\color{outcolor}Out[{\color{outcolor}30}]:} <matplotlib.image.AxesImage at 0x1a818393ba8>
\end{Verbatim}
            
    \begin{center}
    \adjustimage{max size={0.9\linewidth}{0.9\paperheight}}{output_5_2.png}
    \end{center}
    { \hspace*{\fill} \\}
    
    \begin{Verbatim}[commandchars=\\\{\}]
{\color{incolor}In [{\color{incolor}53}]:} \PY{c+c1}{\PYZsh{} ABSOLUTE SOBEL}
         \PY{c+c1}{\PYZsh{} Output is generated for a given kernel and orientation, along with arbitrary thresholds that will be tuned}
         \PY{k}{def} \PY{n+nf}{abs\PYZus{}sobel\PYZus{}thresh}\PY{p}{(}\PY{n}{img}\PY{p}{,} \PY{n}{orient}\PY{o}{=}\PY{l+s+s1}{\PYZsq{}}\PY{l+s+s1}{x}\PY{l+s+s1}{\PYZsq{}}\PY{p}{,} \PY{n}{sobel\PYZus{}kernel}\PY{o}{=}\PY{l+m+mi}{3}\PY{p}{,} \PY{n}{thresh}\PY{o}{=}\PY{p}{(}\PY{l+m+mi}{0}\PY{p}{,} \PY{l+m+mi}{255}\PY{p}{)}\PY{p}{)}\PY{p}{:}
             \PY{c+c1}{\PYZsh{} Convert to grayscale}
             \PY{n}{gray} \PY{o}{=} \PY{n}{cv2}\PY{o}{.}\PY{n}{cvtColor}\PY{p}{(}\PY{n}{img}\PY{p}{,} \PY{n}{cv2}\PY{o}{.}\PY{n}{COLOR\PYZus{}RGB2GRAY}\PY{p}{)}
             \PY{c+c1}{\PYZsh{} Apply x or y gradient with the OpenCV Sobel() function}
             \PY{c+c1}{\PYZsh{} and take the absolute value}
             \PY{k}{if} \PY{n}{orient} \PY{o}{==} \PY{l+s+s1}{\PYZsq{}}\PY{l+s+s1}{x}\PY{l+s+s1}{\PYZsq{}}\PY{p}{:}
                 \PY{n}{abs\PYZus{}sobel} \PY{o}{=} \PY{n}{np}\PY{o}{.}\PY{n}{absolute}\PY{p}{(}\PY{n}{cv2}\PY{o}{.}\PY{n}{Sobel}\PY{p}{(}\PY{n}{gray}\PY{p}{,} \PY{n}{cv2}\PY{o}{.}\PY{n}{CV\PYZus{}64F}\PY{p}{,} \PY{l+m+mi}{1}\PY{p}{,} \PY{l+m+mi}{0}\PY{p}{,} \PY{n}{ksize}\PY{o}{=}\PY{n}{sobel\PYZus{}kernel}\PY{p}{)}\PY{p}{)}
             \PY{k}{if} \PY{n}{orient} \PY{o}{==} \PY{l+s+s1}{\PYZsq{}}\PY{l+s+s1}{y}\PY{l+s+s1}{\PYZsq{}}\PY{p}{:}
                 \PY{n}{abs\PYZus{}sobel} \PY{o}{=} \PY{n}{np}\PY{o}{.}\PY{n}{absolute}\PY{p}{(}\PY{n}{cv2}\PY{o}{.}\PY{n}{Sobel}\PY{p}{(}\PY{n}{gray}\PY{p}{,} \PY{n}{cv2}\PY{o}{.}\PY{n}{CV\PYZus{}64F}\PY{p}{,} \PY{l+m+mi}{0}\PY{p}{,} \PY{l+m+mi}{1}\PY{p}{,} \PY{n}{ksize}\PY{o}{=}\PY{n}{sobel\PYZus{}kernel}\PY{p}{)}\PY{p}{)}
             \PY{c+c1}{\PYZsh{} Rescale back to 8 bit integer}
             \PY{n}{scaled\PYZus{}sobel} \PY{o}{=} \PY{n}{np}\PY{o}{.}\PY{n}{uint8}\PY{p}{(}\PY{l+m+mi}{255}\PY{o}{*}\PY{n}{abs\PYZus{}sobel}\PY{o}{/}\PY{n}{np}\PY{o}{.}\PY{n}{max}\PY{p}{(}\PY{n}{abs\PYZus{}sobel}\PY{p}{)}\PY{p}{)}
             \PY{c+c1}{\PYZsh{} Create a copy and apply the threshold}
             \PY{n}{binary\PYZus{}output} \PY{o}{=} \PY{n}{np}\PY{o}{.}\PY{n}{zeros\PYZus{}like}\PY{p}{(}\PY{n}{scaled\PYZus{}sobel}\PY{p}{)}
             \PY{c+c1}{\PYZsh{} Here I\PYZsq{}m using inclusive (\PYZgt{}=, \PYZlt{}=) thresholds, but exclusive is ok too}
             \PY{n}{binary\PYZus{}output}\PY{p}{[}\PY{p}{(}\PY{n}{scaled\PYZus{}sobel} \PY{o}{\PYZgt{}}\PY{o}{=} \PY{n}{thresh}\PY{p}{[}\PY{l+m+mi}{0}\PY{p}{]}\PY{p}{)} \PY{o}{\PYZam{}} \PY{p}{(}\PY{n}{scaled\PYZus{}sobel} \PY{o}{\PYZlt{}}\PY{o}{=} \PY{n}{thresh}\PY{p}{[}\PY{l+m+mi}{1}\PY{p}{]}\PY{p}{)}\PY{p}{]} \PY{o}{=} \PY{l+m+mi}{1}
         
             \PY{c+c1}{\PYZsh{} Return the result}
             \PY{k}{return} \PY{n}{binary\PYZus{}output}
         
         
         \PY{c+c1}{\PYZsh{} Run the function}
         \PY{n}{grad\PYZus{}binary} \PY{o}{=} \PY{n}{abs\PYZus{}sobel\PYZus{}thresh}\PY{p}{(}\PY{n}{image}\PY{p}{,} \PY{n}{orient}\PY{o}{=}\PY{l+s+s1}{\PYZsq{}}\PY{l+s+s1}{x}\PY{l+s+s1}{\PYZsq{}}\PY{p}{,} \PY{n}{sobel\PYZus{}kernel}\PY{o}{=}\PY{l+m+mi}{3}\PY{p}{,} \PY{n}{thresh}\PY{o}{=}\PY{p}{(}\PY{l+m+mi}{20}\PY{p}{,} \PY{l+m+mi}{200}\PY{p}{)}\PY{p}{)}
         \PY{c+c1}{\PYZsh{} Plot the result}
         \PY{n}{f}\PY{p}{,} \PY{p}{(}\PY{n}{ax1}\PY{p}{,} \PY{n}{ax2}\PY{p}{)} \PY{o}{=} \PY{n}{plt}\PY{o}{.}\PY{n}{subplots}\PY{p}{(}\PY{l+m+mi}{1}\PY{p}{,} \PY{l+m+mi}{2}\PY{p}{,} \PY{n}{figsize}\PY{o}{=}\PY{p}{(}\PY{l+m+mi}{24}\PY{p}{,} \PY{l+m+mi}{9}\PY{p}{)}\PY{p}{)}
         \PY{n}{f}\PY{o}{.}\PY{n}{tight\PYZus{}layout}\PY{p}{(}\PY{p}{)}
         \PY{n}{ax1}\PY{o}{.}\PY{n}{imshow}\PY{p}{(}\PY{n}{image}\PY{p}{)}
         \PY{n}{ax1}\PY{o}{.}\PY{n}{set\PYZus{}title}\PY{p}{(}\PY{l+s+s1}{\PYZsq{}}\PY{l+s+s1}{Original Image}\PY{l+s+s1}{\PYZsq{}}\PY{p}{,} \PY{n}{fontsize}\PY{o}{=}\PY{l+m+mi}{50}\PY{p}{)}
         \PY{n}{ax2}\PY{o}{.}\PY{n}{imshow}\PY{p}{(}\PY{n}{grad\PYZus{}binary}\PY{p}{,} \PY{n}{cmap}\PY{o}{=}\PY{l+s+s1}{\PYZsq{}}\PY{l+s+s1}{gray}\PY{l+s+s1}{\PYZsq{}}\PY{p}{)}
         \PY{n}{ax2}\PY{o}{.}\PY{n}{set\PYZus{}title}\PY{p}{(}\PY{l+s+s1}{\PYZsq{}}\PY{l+s+s1}{Thresholded Gradient}\PY{l+s+s1}{\PYZsq{}}\PY{p}{,} \PY{n}{fontsize}\PY{o}{=}\PY{l+m+mi}{50}\PY{p}{)}
         \PY{n}{plt}\PY{o}{.}\PY{n}{subplots\PYZus{}adjust}\PY{p}{(}\PY{n}{left}\PY{o}{=}\PY{l+m+mf}{0.}\PY{p}{,} \PY{n}{right}\PY{o}{=}\PY{l+m+mi}{1}\PY{p}{,} \PY{n}{top}\PY{o}{=}\PY{l+m+mf}{0.9}\PY{p}{,} \PY{n}{bottom}\PY{o}{=}\PY{l+m+mf}{0.}\PY{p}{)}
\end{Verbatim}


    \begin{center}
    \adjustimage{max size={0.9\linewidth}{0.9\paperheight}}{output_6_0.png}
    \end{center}
    { \hspace*{\fill} \\}
    
    \begin{Verbatim}[commandchars=\\\{\}]
{\color{incolor}In [{\color{incolor}48}]:} \PY{c+c1}{\PYZsh{} DIRECTION SOBEL}
         \PY{c+c1}{\PYZsh{} Producing the outline based on the kernel and the angular direction}
         \PY{k}{def} \PY{n+nf}{dir\PYZus{}threshold}\PY{p}{(}\PY{n}{img}\PY{p}{,} \PY{n}{sobel\PYZus{}kernel}\PY{o}{=}\PY{l+m+mi}{3}\PY{p}{,} \PY{n}{thresh}\PY{o}{=}\PY{p}{(}\PY{l+m+mi}{0}\PY{p}{,} \PY{n}{np}\PY{o}{.}\PY{n}{pi}\PY{o}{/}\PY{l+m+mi}{2}\PY{p}{)}\PY{p}{)}\PY{p}{:}
             \PY{c+c1}{\PYZsh{} Grayscale}
             \PY{n}{gray} \PY{o}{=} \PY{n}{cv2}\PY{o}{.}\PY{n}{cvtColor}\PY{p}{(}\PY{n}{img}\PY{p}{,} \PY{n}{cv2}\PY{o}{.}\PY{n}{COLOR\PYZus{}RGB2GRAY}\PY{p}{)}
             \PY{c+c1}{\PYZsh{} Calculate the x and y gradients}
             \PY{n}{sobelx} \PY{o}{=} \PY{n}{cv2}\PY{o}{.}\PY{n}{Sobel}\PY{p}{(}\PY{n}{gray}\PY{p}{,} \PY{n}{cv2}\PY{o}{.}\PY{n}{CV\PYZus{}64F}\PY{p}{,} \PY{l+m+mi}{1}\PY{p}{,} \PY{l+m+mi}{0}\PY{p}{,} \PY{n}{ksize}\PY{o}{=}\PY{n}{sobel\PYZus{}kernel}\PY{p}{)}
             \PY{n}{sobely} \PY{o}{=} \PY{n}{cv2}\PY{o}{.}\PY{n}{Sobel}\PY{p}{(}\PY{n}{gray}\PY{p}{,} \PY{n}{cv2}\PY{o}{.}\PY{n}{CV\PYZus{}64F}\PY{p}{,} \PY{l+m+mi}{0}\PY{p}{,} \PY{l+m+mi}{1}\PY{p}{,} \PY{n}{ksize}\PY{o}{=}\PY{n}{sobel\PYZus{}kernel}\PY{p}{)}
             \PY{c+c1}{\PYZsh{} Take the absolute value of the gradient direction, }
             \PY{c+c1}{\PYZsh{} apply a threshold, and create a binary image result}
             \PY{n}{absgraddir} \PY{o}{=} \PY{n}{np}\PY{o}{.}\PY{n}{arctan2}\PY{p}{(}\PY{n}{np}\PY{o}{.}\PY{n}{absolute}\PY{p}{(}\PY{n}{sobely}\PY{p}{)}\PY{p}{,} \PY{n}{np}\PY{o}{.}\PY{n}{absolute}\PY{p}{(}\PY{n}{sobelx}\PY{p}{)}\PY{p}{)}
             \PY{n}{binary\PYZus{}output} \PY{o}{=}  \PY{n}{np}\PY{o}{.}\PY{n}{zeros\PYZus{}like}\PY{p}{(}\PY{n}{absgraddir}\PY{p}{)}
             \PY{n}{binary\PYZus{}output}\PY{p}{[}\PY{p}{(}\PY{n}{absgraddir} \PY{o}{\PYZgt{}}\PY{o}{=} \PY{n}{thresh}\PY{p}{[}\PY{l+m+mi}{0}\PY{p}{]}\PY{p}{)} \PY{o}{\PYZam{}} \PY{p}{(}\PY{n}{absgraddir} \PY{o}{\PYZlt{}}\PY{o}{=} \PY{n}{thresh}\PY{p}{[}\PY{l+m+mi}{1}\PY{p}{]}\PY{p}{)}\PY{p}{]} \PY{o}{=} \PY{l+m+mi}{1}
         
             \PY{c+c1}{\PYZsh{} Return the binary image}
             \PY{k}{return} \PY{n}{binary\PYZus{}output}
         
         \PY{n}{dir\PYZus{}binary} \PY{o}{=} \PY{n}{dir\PYZus{}threshold}\PY{p}{(}\PY{n}{image}\PY{p}{,} \PY{n}{sobel\PYZus{}kernel}\PY{o}{=}\PY{l+m+mi}{15}\PY{p}{,} \PY{n}{thresh}\PY{o}{=}\PY{p}{(}\PY{l+m+mf}{0.7}\PY{p}{,} \PY{l+m+mf}{1.0}\PY{p}{)}\PY{p}{)}
         \PY{c+c1}{\PYZsh{} Plot the result}
         \PY{n}{f}\PY{p}{,} \PY{p}{(}\PY{n}{ax1}\PY{p}{,} \PY{n}{ax2}\PY{p}{)} \PY{o}{=} \PY{n}{plt}\PY{o}{.}\PY{n}{subplots}\PY{p}{(}\PY{l+m+mi}{1}\PY{p}{,} \PY{l+m+mi}{2}\PY{p}{,} \PY{n}{figsize}\PY{o}{=}\PY{p}{(}\PY{l+m+mi}{24}\PY{p}{,} \PY{l+m+mi}{9}\PY{p}{)}\PY{p}{)}
         \PY{n}{f}\PY{o}{.}\PY{n}{tight\PYZus{}layout}\PY{p}{(}\PY{p}{)}
         \PY{n}{ax1}\PY{o}{.}\PY{n}{imshow}\PY{p}{(}\PY{n}{image}\PY{p}{)}
         \PY{n}{ax1}\PY{o}{.}\PY{n}{set\PYZus{}title}\PY{p}{(}\PY{l+s+s1}{\PYZsq{}}\PY{l+s+s1}{Original Image}\PY{l+s+s1}{\PYZsq{}}\PY{p}{,} \PY{n}{fontsize}\PY{o}{=}\PY{l+m+mi}{50}\PY{p}{)}
         \PY{n}{ax2}\PY{o}{.}\PY{n}{imshow}\PY{p}{(}\PY{n}{dir\PYZus{}binary}\PY{p}{,} \PY{n}{cmap}\PY{o}{=}\PY{l+s+s1}{\PYZsq{}}\PY{l+s+s1}{gray}\PY{l+s+s1}{\PYZsq{}}\PY{p}{)}
         \PY{n}{ax2}\PY{o}{.}\PY{n}{set\PYZus{}title}\PY{p}{(}\PY{l+s+s1}{\PYZsq{}}\PY{l+s+s1}{Thresholded Grad. Dir.}\PY{l+s+s1}{\PYZsq{}}\PY{p}{,} \PY{n}{fontsize}\PY{o}{=}\PY{l+m+mi}{50}\PY{p}{)}
         \PY{n}{plt}\PY{o}{.}\PY{n}{subplots\PYZus{}adjust}\PY{p}{(}\PY{n}{left}\PY{o}{=}\PY{l+m+mf}{0.}\PY{p}{,} \PY{n}{right}\PY{o}{=}\PY{l+m+mi}{1}\PY{p}{,} \PY{n}{top}\PY{o}{=}\PY{l+m+mf}{0.9}\PY{p}{,} \PY{n}{bottom}\PY{o}{=}\PY{l+m+mf}{0.}\PY{p}{)}
\end{Verbatim}


    \begin{center}
    \adjustimage{max size={0.9\linewidth}{0.9\paperheight}}{output_7_0.png}
    \end{center}
    { \hspace*{\fill} \\}
    
    \begin{Verbatim}[commandchars=\\\{\}]
{\color{incolor}In [{\color{incolor}40}]:} \PY{c+c1}{\PYZsh{} MAGNITUDE SOBEL}
         \PY{k}{def} \PY{n+nf}{mag\PYZus{}thresh}\PY{p}{(}\PY{n}{img}\PY{p}{,} \PY{n}{sobel\PYZus{}kernel}\PY{o}{=}\PY{l+m+mi}{3}\PY{p}{,} \PY{n}{mag\PYZus{}thresh}\PY{o}{=}\PY{p}{(}\PY{l+m+mi}{0}\PY{p}{,} \PY{l+m+mi}{255}\PY{p}{)}\PY{p}{)}\PY{p}{:}
             \PY{c+c1}{\PYZsh{} Convert to grayscale}
             \PY{n}{gray} \PY{o}{=} \PY{n}{cv2}\PY{o}{.}\PY{n}{cvtColor}\PY{p}{(}\PY{n}{img}\PY{p}{,} \PY{n}{cv2}\PY{o}{.}\PY{n}{COLOR\PYZus{}RGB2GRAY}\PY{p}{)}
             \PY{c+c1}{\PYZsh{}plt.imshow(image)}
             
             \PY{c+c1}{\PYZsh{} Take both Sobel x and y gradients}
             \PY{n}{sobelx} \PY{o}{=} \PY{n}{cv2}\PY{o}{.}\PY{n}{Sobel}\PY{p}{(}\PY{n}{gray}\PY{p}{,} \PY{n}{cv2}\PY{o}{.}\PY{n}{CV\PYZus{}64F}\PY{p}{,} \PY{l+m+mi}{1}\PY{p}{,} \PY{l+m+mi}{0}\PY{p}{,} \PY{n}{ksize}\PY{o}{=}\PY{n}{sobel\PYZus{}kernel}\PY{p}{)}
             \PY{n}{sobely} \PY{o}{=} \PY{n}{cv2}\PY{o}{.}\PY{n}{Sobel}\PY{p}{(}\PY{n}{gray}\PY{p}{,} \PY{n}{cv2}\PY{o}{.}\PY{n}{CV\PYZus{}64F}\PY{p}{,} \PY{l+m+mi}{0}\PY{p}{,} \PY{l+m+mi}{1}\PY{p}{,} \PY{n}{ksize}\PY{o}{=}\PY{n}{sobel\PYZus{}kernel}\PY{p}{)}
             \PY{c+c1}{\PYZsh{} Calculate the gradient magnitude}
             \PY{n}{gradmag} \PY{o}{=} \PY{n}{np}\PY{o}{.}\PY{n}{sqrt}\PY{p}{(}\PY{n}{sobelx}\PY{o}{*}\PY{o}{*}\PY{l+m+mi}{2} \PY{o}{+} \PY{n}{sobely}\PY{o}{*}\PY{o}{*}\PY{l+m+mi}{2}\PY{p}{)}
             \PY{c+c1}{\PYZsh{} Rescale to 8 bit}
             \PY{n}{scale\PYZus{}factor} \PY{o}{=} \PY{n}{np}\PY{o}{.}\PY{n}{max}\PY{p}{(}\PY{n}{gradmag}\PY{p}{)}\PY{o}{/}\PY{l+m+mi}{255} 
             \PY{n}{gradmag} \PY{o}{=} \PY{p}{(}\PY{n}{gradmag}\PY{o}{/}\PY{n}{scale\PYZus{}factor}\PY{p}{)}\PY{o}{.}\PY{n}{astype}\PY{p}{(}\PY{n}{np}\PY{o}{.}\PY{n}{uint8}\PY{p}{)} 
             \PY{c+c1}{\PYZsh{} Create a binary image of ones where threshold is met, zeros otherwise}
             \PY{n}{binary\PYZus{}output} \PY{o}{=} \PY{n}{np}\PY{o}{.}\PY{n}{zeros\PYZus{}like}\PY{p}{(}\PY{n}{gradmag}\PY{p}{)}
             \PY{n}{binary\PYZus{}output}\PY{p}{[}\PY{p}{(}\PY{n}{gradmag} \PY{o}{\PYZgt{}}\PY{o}{=} \PY{n}{mag\PYZus{}thresh}\PY{p}{[}\PY{l+m+mi}{0}\PY{p}{]}\PY{p}{)} \PY{o}{\PYZam{}} \PY{p}{(}\PY{n}{gradmag} \PY{o}{\PYZlt{}}\PY{o}{=} \PY{n}{mag\PYZus{}thresh}\PY{p}{[}\PY{l+m+mi}{1}\PY{p}{]}\PY{p}{)}\PY{p}{]} \PY{o}{=} \PY{l+m+mi}{1}
         
             \PY{c+c1}{\PYZsh{} Return the binary image}
             \PY{k}{return} \PY{n}{binary\PYZus{}output}
         
         \PY{n}{mag\PYZus{}binary} \PY{o}{=} \PY{n}{mag\PYZus{}thresh}\PY{p}{(}\PY{n}{image}\PY{p}{,} \PY{n}{sobel\PYZus{}kernel}\PY{o}{=}\PY{l+m+mi}{3}\PY{p}{,} \PY{n}{mag\PYZus{}thresh}\PY{o}{=}\PY{p}{(}\PY{l+m+mi}{30}\PY{p}{,} \PY{l+m+mi}{100}\PY{p}{)}\PY{p}{)}
         \PY{c+c1}{\PYZsh{} Plot the result}
         \PY{n}{f}\PY{p}{,} \PY{p}{(}\PY{n}{ax1}\PY{p}{,} \PY{n}{ax2}\PY{p}{)} \PY{o}{=} \PY{n}{plt}\PY{o}{.}\PY{n}{subplots}\PY{p}{(}\PY{l+m+mi}{1}\PY{p}{,} \PY{l+m+mi}{2}\PY{p}{,} \PY{n}{figsize}\PY{o}{=}\PY{p}{(}\PY{l+m+mi}{24}\PY{p}{,} \PY{l+m+mi}{9}\PY{p}{)}\PY{p}{)}
         \PY{n}{f}\PY{o}{.}\PY{n}{tight\PYZus{}layout}\PY{p}{(}\PY{p}{)}
         \PY{n}{ax1}\PY{o}{.}\PY{n}{imshow}\PY{p}{(}\PY{n}{image}\PY{p}{)}
         \PY{n}{ax1}\PY{o}{.}\PY{n}{set\PYZus{}title}\PY{p}{(}\PY{l+s+s1}{\PYZsq{}}\PY{l+s+s1}{Original Image}\PY{l+s+s1}{\PYZsq{}}\PY{p}{,} \PY{n}{fontsize}\PY{o}{=}\PY{l+m+mi}{50}\PY{p}{)}
         \PY{n}{ax2}\PY{o}{.}\PY{n}{imshow}\PY{p}{(}\PY{n}{mag\PYZus{}binary}\PY{p}{,} \PY{n}{cmap}\PY{o}{=}\PY{l+s+s1}{\PYZsq{}}\PY{l+s+s1}{gray}\PY{l+s+s1}{\PYZsq{}}\PY{p}{)}
         \PY{n}{ax2}\PY{o}{.}\PY{n}{set\PYZus{}title}\PY{p}{(}\PY{l+s+s1}{\PYZsq{}}\PY{l+s+s1}{Thresholded Magnitude}\PY{l+s+s1}{\PYZsq{}}\PY{p}{,} \PY{n}{fontsize}\PY{o}{=}\PY{l+m+mi}{50}\PY{p}{)}
         \PY{n}{plt}\PY{o}{.}\PY{n}{subplots\PYZus{}adjust}\PY{p}{(}\PY{n}{left}\PY{o}{=}\PY{l+m+mf}{0.}\PY{p}{,} \PY{n}{right}\PY{o}{=}\PY{l+m+mi}{1}\PY{p}{,} \PY{n}{top}\PY{o}{=}\PY{l+m+mf}{0.9}\PY{p}{,} \PY{n}{bottom}\PY{o}{=}\PY{l+m+mf}{0.}\PY{p}{)}
\end{Verbatim}


    \begin{center}
    \adjustimage{max size={0.9\linewidth}{0.9\paperheight}}{output_8_0.png}
    \end{center}
    { \hspace*{\fill} \\}
    
    \begin{Verbatim}[commandchars=\\\{\}]
{\color{incolor}In [{\color{incolor}60}]:} \PY{c+c1}{\PYZsh{} COLOR Thresholding}
         \PY{c+c1}{\PYZsh{} Re\PYZhy{}using the function created in lesson 11 HLS Quiz}
         \PY{c+c1}{\PYZsh{} Focus on the Saturation channel of the HLS}
         \PY{k}{def} \PY{n+nf}{hls\PYZus{}select}\PY{p}{(}\PY{n}{img}\PY{p}{,} \PY{n}{thresh}\PY{o}{=}\PY{p}{(}\PY{l+m+mi}{0}\PY{p}{,} \PY{l+m+mi}{255}\PY{p}{)}\PY{p}{)}\PY{p}{:}
             \PY{n}{hls} \PY{o}{=} \PY{n}{cv2}\PY{o}{.}\PY{n}{cvtColor}\PY{p}{(}\PY{n}{img}\PY{p}{,} \PY{n}{cv2}\PY{o}{.}\PY{n}{COLOR\PYZus{}RGB2HLS}\PY{p}{)}
             \PY{n}{s\PYZus{}channel} \PY{o}{=} \PY{n}{hls}\PY{p}{[}\PY{p}{:}\PY{p}{,}\PY{p}{:}\PY{p}{,}\PY{l+m+mi}{2}\PY{p}{]}
             \PY{n}{binary\PYZus{}output} \PY{o}{=} \PY{n}{np}\PY{o}{.}\PY{n}{zeros\PYZus{}like}\PY{p}{(}\PY{n}{s\PYZus{}channel}\PY{p}{)}
             \PY{n}{binary\PYZus{}output}\PY{p}{[}\PY{p}{(}\PY{n}{s\PYZus{}channel} \PY{o}{\PYZgt{}} \PY{n}{thresh}\PY{p}{[}\PY{l+m+mi}{0}\PY{p}{]}\PY{p}{)} \PY{o}{\PYZam{}} \PY{p}{(}\PY{n}{s\PYZus{}channel} \PY{o}{\PYZlt{}}\PY{o}{=} \PY{n}{thresh}\PY{p}{[}\PY{l+m+mi}{1}\PY{p}{]}\PY{p}{)}\PY{p}{]} \PY{o}{=} \PY{l+m+mi}{1}
             \PY{k}{return} \PY{n}{binary\PYZus{}output}
         
         \PY{n}{hls\PYZus{}binary} \PY{o}{=} \PY{n}{hls\PYZus{}select}\PY{p}{(}\PY{n}{image}\PY{p}{,} \PY{n}{thresh}\PY{o}{=}\PY{p}{(}\PY{l+m+mi}{90}\PY{p}{,} \PY{l+m+mi}{255}\PY{p}{)}\PY{p}{)}
         \PY{n}{plt}\PY{o}{.}\PY{n}{imshow}\PY{p}{(}\PY{n}{hls\PYZus{}binary}\PY{p}{,} \PY{n}{cmap}\PY{o}{=}\PY{l+s+s1}{\PYZsq{}}\PY{l+s+s1}{gray}\PY{l+s+s1}{\PYZsq{}}\PY{p}{)}
         \PY{n}{plt}\PY{o}{.}\PY{n}{show}\PY{p}{(}\PY{p}{)}
\end{Verbatim}


    \begin{center}
    \adjustimage{max size={0.9\linewidth}{0.9\paperheight}}{output_9_0.png}
    \end{center}
    { \hspace*{\fill} \\}
    
    \begin{Verbatim}[commandchars=\\\{\}]
{\color{incolor}In [{\color{incolor}69}]:} \PY{c+c1}{\PYZsh{} COMBINATION of Sobel transforms and the COLOR thresholding}
         \PY{n}{ksize} \PY{o}{=} \PY{l+m+mi}{3}
         \PY{n}{gradx} \PY{o}{=} \PY{n}{abs\PYZus{}sobel\PYZus{}thresh}\PY{p}{(}\PY{n}{image}\PY{p}{,} \PY{n}{orient}\PY{o}{=}\PY{l+s+s1}{\PYZsq{}}\PY{l+s+s1}{x}\PY{l+s+s1}{\PYZsq{}}\PY{p}{,} \PY{n}{sobel\PYZus{}kernel}\PY{o}{=}\PY{n}{ksize}\PY{p}{,} \PY{n}{thresh}\PY{o}{=}\PY{p}{(}\PY{l+m+mi}{20}\PY{p}{,} \PY{l+m+mi}{100}\PY{p}{)}\PY{p}{)}
         \PY{n}{grady} \PY{o}{=} \PY{n}{abs\PYZus{}sobel\PYZus{}thresh}\PY{p}{(}\PY{n}{image}\PY{p}{,} \PY{n}{orient}\PY{o}{=}\PY{l+s+s1}{\PYZsq{}}\PY{l+s+s1}{y}\PY{l+s+s1}{\PYZsq{}}\PY{p}{,} \PY{n}{sobel\PYZus{}kernel}\PY{o}{=}\PY{n}{ksize}\PY{p}{,} \PY{n}{thresh}\PY{o}{=}\PY{p}{(}\PY{l+m+mi}{20}\PY{p}{,} \PY{l+m+mi}{100}\PY{p}{)}\PY{p}{)}
         \PY{n}{mag\PYZus{}binary} \PY{o}{=} \PY{n}{mag\PYZus{}thresh}\PY{p}{(}\PY{n}{image}\PY{p}{,} \PY{n}{sobel\PYZus{}kernel}\PY{o}{=}\PY{n}{ksize}\PY{p}{,} \PY{n}{mag\PYZus{}thresh}\PY{o}{=}\PY{p}{(}\PY{l+m+mi}{30}\PY{p}{,} \PY{l+m+mi}{100}\PY{p}{)}\PY{p}{)}
         \PY{n}{dir\PYZus{}binary} \PY{o}{=} \PY{n}{dir\PYZus{}threshold}\PY{p}{(}\PY{n}{image}\PY{p}{,} \PY{n}{sobel\PYZus{}kernel}\PY{o}{=}\PY{n}{ksize}\PY{p}{,} \PY{n}{thresh}\PY{o}{=}\PY{p}{(}\PY{l+m+mf}{0.7}\PY{p}{,} \PY{l+m+mf}{1.0}\PY{p}{)}\PY{p}{)}
         \PY{c+c1}{\PYZsh{} COLOR Thresholding}
         \PY{n}{hls\PYZus{}binary} \PY{o}{=} \PY{n}{hls\PYZus{}select}\PY{p}{(}\PY{n}{image}\PY{p}{,} \PY{n}{thresh}\PY{o}{=}\PY{p}{(}\PY{l+m+mi}{90}\PY{p}{,} \PY{l+m+mi}{255}\PY{p}{)}\PY{p}{)}
         
         \PY{c+c1}{\PYZsh{} Original thresholds}
         \PY{c+c1}{\PYZsh{}(20, 100)}
         \PY{c+c1}{\PYZsh{}(30, 100)}
         \PY{c+c1}{\PYZsh{}(0.7, 1.3)}
         \PY{n}{combined} \PY{o}{=} \PY{n}{np}\PY{o}{.}\PY{n}{zeros\PYZus{}like}\PY{p}{(}\PY{n}{dir\PYZus{}binary}\PY{p}{)}
         \PY{c+c1}{\PYZsh{}combined[ ((gradx == 1) \PYZam{} (grady == 1)) | ((mag\PYZus{}binary == 1) \PYZam{} (dir\PYZus{}binary == 1))  | hls\PYZus{}binary == 1 ] = 1}
         
         \PY{n}{combined}\PY{p}{[}  \PY{p}{(}\PY{p}{(}\PY{n}{dir\PYZus{}binary} \PY{o}{==} \PY{l+m+mi}{1}\PY{p}{)} \PY{o}{\PYZam{}}\PY{p}{(}\PY{n}{mag\PYZus{}binary} \PY{o}{==} \PY{l+m+mi}{1}\PY{p}{)} \PY{p}{)}  \PY{o}{|} \PY{n}{hls\PYZus{}binary} \PY{o}{==} \PY{l+m+mi}{1} \PY{p}{]} \PY{o}{=} \PY{l+m+mi}{1}
         
         
         \PY{n}{f}\PY{p}{,} \PY{p}{(}\PY{n}{ax1}\PY{p}{,} \PY{n}{ax2}\PY{p}{)} \PY{o}{=} \PY{n}{plt}\PY{o}{.}\PY{n}{subplots}\PY{p}{(}\PY{l+m+mi}{1}\PY{p}{,} \PY{l+m+mi}{2}\PY{p}{,} \PY{n}{figsize}\PY{o}{=}\PY{p}{(}\PY{l+m+mi}{24}\PY{p}{,} \PY{l+m+mi}{9}\PY{p}{)}\PY{p}{)}
         \PY{n}{f}\PY{o}{.}\PY{n}{tight\PYZus{}layout}\PY{p}{(}\PY{p}{)}
         \PY{n}{ax1}\PY{o}{.}\PY{n}{imshow}\PY{p}{(}\PY{n}{image}\PY{p}{)}
         \PY{n}{ax1}\PY{o}{.}\PY{n}{set\PYZus{}title}\PY{p}{(}\PY{l+s+s1}{\PYZsq{}}\PY{l+s+s1}{Original Image}\PY{l+s+s1}{\PYZsq{}}\PY{p}{,} \PY{n}{fontsize}\PY{o}{=}\PY{l+m+mi}{50}\PY{p}{)}
         \PY{n}{ax2}\PY{o}{.}\PY{n}{imshow}\PY{p}{(}\PY{n}{combined}\PY{p}{,} \PY{n}{cmap}\PY{o}{=}\PY{l+s+s1}{\PYZsq{}}\PY{l+s+s1}{gray}\PY{l+s+s1}{\PYZsq{}}\PY{p}{)}
         \PY{n}{ax2}\PY{o}{.}\PY{n}{set\PYZus{}title}\PY{p}{(}\PY{l+s+s1}{\PYZsq{}}\PY{l+s+s1}{Combined Thresholds}\PY{l+s+s1}{\PYZsq{}}\PY{p}{,} \PY{n}{fontsize}\PY{o}{=}\PY{l+m+mi}{50}\PY{p}{)}
         \PY{n}{plt}\PY{o}{.}\PY{n}{subplots\PYZus{}adjust}\PY{p}{(}\PY{n}{left}\PY{o}{=}\PY{l+m+mf}{0.}\PY{p}{,} \PY{n}{right}\PY{o}{=}\PY{l+m+mi}{1}\PY{p}{,} \PY{n}{top}\PY{o}{=}\PY{l+m+mf}{0.9}\PY{p}{,} \PY{n}{bottom}\PY{o}{=}\PY{l+m+mf}{0.}\PY{p}{)}
         
         \PY{c+c1}{\PYZsh{}plt.imshow(grady, cmap=\PYZsq{}gray\PYZsq{})}
         \PY{c+c1}{\PYZsh{}plt.show()}
         \PY{c+c1}{\PYZsh{}plt.imshow(gradx, cmap=\PYZsq{}gray\PYZsq{})}
         \PY{c+c1}{\PYZsh{}plt.show()}
\end{Verbatim}


    \begin{center}
    \adjustimage{max size={0.9\linewidth}{0.9\paperheight}}{output_10_0.png}
    \end{center}
    { \hspace*{\fill} \\}
    
    \begin{Verbatim}[commandchars=\\\{\}]
{\color{incolor}In [{\color{incolor} }]:} \PY{c+c1}{\PYZsh{} Maybe first dirbinary?}
        
        \PY{n}{dir\PYZus{}binary} \PY{o}{=} \PY{n}{dir\PYZus{}threshold}\PY{p}{(}\PY{n}{image}\PY{p}{,} \PY{n}{sobel\PYZus{}kernel}\PY{o}{=}\PY{n}{ksize}\PY{p}{,} \PY{n}{thresh}\PY{o}{=}\PY{p}{(}\PY{l+m+mf}{0.7}\PY{p}{,} \PY{l+m+mf}{1.0}\PY{p}{)}\PY{p}{)}
        \PY{n}{gradx} \PY{o}{=} \PY{n}{abs\PYZus{}sobel\PYZus{}thresh}\PY{p}{(}\PY{n}{image}\PY{p}{,} \PY{n}{orient}\PY{o}{=}\PY{l+s+s1}{\PYZsq{}}\PY{l+s+s1}{x}\PY{l+s+s1}{\PYZsq{}}\PY{p}{,} \PY{n}{sobel\PYZus{}kernel}\PY{o}{=}\PY{n}{ksize}\PY{p}{,} \PY{n}{thresh}\PY{o}{=}\PY{p}{(}\PY{l+m+mi}{20}\PY{p}{,} \PY{l+m+mi}{100}\PY{p}{)}\PY{p}{)}
\end{Verbatim}


    \begin{Verbatim}[commandchars=\\\{\}]
{\color{incolor}In [{\color{incolor}6}]:} \PY{c+c1}{\PYZsh{} Change to perspective change ONLY!! \PYZhy{} we undistort already in the second step of the project}
        
        \PY{c+c1}{\PYZsh{} The function performing undistortion and transformation}
        \PY{c+c1}{\PYZsh{} The function is mostly reused from lesson 18. Undistort and Transform}
        \PY{k}{def} \PY{n+nf}{corners\PYZus{}unwarp}\PY{p}{(}\PY{n}{img}\PY{p}{,} \PY{n}{nx}\PY{p}{,} \PY{n}{ny}\PY{p}{,} \PY{n}{mtx}\PY{p}{,} \PY{n}{dist}\PY{p}{)}\PY{p}{:}
            \PY{c+c1}{\PYZsh{} Removing distortion}
            \PY{n}{undist} \PY{o}{=} \PY{n}{cv2}\PY{o}{.}\PY{n}{undistort}\PY{p}{(}\PY{n}{img}\PY{p}{,} \PY{n}{mtx}\PY{p}{,} \PY{n}{dist}\PY{p}{,} \PY{k+kc}{None}\PY{p}{,} \PY{n}{mtx}\PY{p}{)}
            \PY{c+c1}{\PYZsh{} Converting to grayscale}
            \PY{n}{gray} \PY{o}{=} \PY{n}{cv2}\PY{o}{.}\PY{n}{cvtColor}\PY{p}{(}\PY{n}{undist}\PY{p}{,} \PY{n}{cv2}\PY{o}{.}\PY{n}{COLOR\PYZus{}BGR2GRAY}\PY{p}{)}
            \PY{c+c1}{\PYZsh{} Identifying corners (number of corners is predefined in the function call)}
            \PY{n}{ret}\PY{p}{,} \PY{n}{corners} \PY{o}{=} \PY{n}{cv2}\PY{o}{.}\PY{n}{findChessboardCorners}\PY{p}{(}\PY{n}{gray}\PY{p}{,} \PY{p}{(}\PY{n}{nx}\PY{p}{,} \PY{n}{ny}\PY{p}{)}\PY{p}{,} \PY{k+kc}{None}\PY{p}{)}
        
            \PY{k}{if} \PY{n}{ret} \PY{o}{==} \PY{k+kc}{True}\PY{p}{:}
                \PY{n}{cv2}\PY{o}{.}\PY{n}{drawChessboardCorners}\PY{p}{(}\PY{n}{undist}\PY{p}{,} \PY{p}{(}\PY{n}{nx}\PY{p}{,} \PY{n}{ny}\PY{p}{)}\PY{p}{,} \PY{n}{corners}\PY{p}{,} \PY{n}{ret}\PY{p}{)}
                \PY{c+c1}{\PYZsh{} Arbitrary offset for cropping to preserve the aspect ratio}
                \PY{n}{offset} \PY{o}{=} \PY{l+m+mi}{100} 
                \PY{n}{img\PYZus{}size} \PY{o}{=} \PY{p}{(}\PY{n}{gray}\PY{o}{.}\PY{n}{shape}\PY{p}{[}\PY{l+m+mi}{1}\PY{p}{]}\PY{p}{,} \PY{n}{gray}\PY{o}{.}\PY{n}{shape}\PY{p}{[}\PY{l+m+mi}{0}\PY{p}{]}\PY{p}{)}
        
                \PY{c+c1}{\PYZsh{} Identifying the fringe corners}
                \PY{n}{src} \PY{o}{=} \PY{n}{np}\PY{o}{.}\PY{n}{float32}\PY{p}{(}\PY{p}{[}\PY{n}{corners}\PY{p}{[}\PY{l+m+mi}{0}\PY{p}{]}\PY{p}{,} \PY{n}{corners}\PY{p}{[}\PY{n}{nx}\PY{o}{\PYZhy{}}\PY{l+m+mi}{1}\PY{p}{]}\PY{p}{,} \PY{n}{corners}\PY{p}{[}\PY{o}{\PYZhy{}}\PY{l+m+mi}{1}\PY{p}{]}\PY{p}{,} \PY{n}{corners}\PY{p}{[}\PY{o}{\PYZhy{}}\PY{n}{nx}\PY{p}{]}\PY{p}{]}\PY{p}{)}
                \PY{c+c1}{\PYZsh{} Arbitrary four points based on offset crop}
                \PY{n}{dst} \PY{o}{=} \PY{n}{np}\PY{o}{.}\PY{n}{float32}\PY{p}{(}\PY{p}{[}\PY{p}{[}\PY{n}{offset}\PY{p}{,} \PY{n}{offset}\PY{p}{]}\PY{p}{,} \PY{p}{[}\PY{n}{img\PYZus{}size}\PY{p}{[}\PY{l+m+mi}{0}\PY{p}{]}\PY{o}{\PYZhy{}}\PY{n}{offset}\PY{p}{,} \PY{n}{offset}\PY{p}{]}\PY{p}{,} 
                                             \PY{p}{[}\PY{n}{img\PYZus{}size}\PY{p}{[}\PY{l+m+mi}{0}\PY{p}{]}\PY{o}{\PYZhy{}}\PY{n}{offset}\PY{p}{,} \PY{n}{img\PYZus{}size}\PY{p}{[}\PY{l+m+mi}{1}\PY{p}{]}\PY{o}{\PYZhy{}}\PY{n}{offset}\PY{p}{]}\PY{p}{,} 
                                             \PY{p}{[}\PY{n}{offset}\PY{p}{,} \PY{n}{img\PYZus{}size}\PY{p}{[}\PY{l+m+mi}{1}\PY{p}{]}\PY{o}{\PYZhy{}}\PY{n}{offset}\PY{p}{]}\PY{p}{]}\PY{p}{)}
                \PY{c+c1}{\PYZsh{} Extracting perspective transform matrix}
                \PY{n}{M} \PY{o}{=} \PY{n}{cv2}\PY{o}{.}\PY{n}{getPerspectiveTransform}\PY{p}{(}\PY{n}{src}\PY{p}{,} \PY{n}{dst}\PY{p}{)}
                \PY{c+c1}{\PYZsh{} Warping the image}
                \PY{n}{warped} \PY{o}{=} \PY{n}{cv2}\PY{o}{.}\PY{n}{warpPerspective}\PY{p}{(}\PY{n}{undist}\PY{p}{,} \PY{n}{M}\PY{p}{,} \PY{n}{img\PYZus{}size}\PY{p}{)}
        
            \PY{c+c1}{\PYZsh{} Return the resulting image and matrix}
            \PY{k}{return} \PY{n}{warped}\PY{p}{,} \PY{n}{M}
        
        \PY{c+c1}{\PYZsh{} Testing the function on the sample image}
        \PY{n}{nx} \PY{o}{=} \PY{l+m+mi}{9}
        \PY{n}{ny} \PY{o}{=} \PY{l+m+mi}{6}
        \PY{n}{img} \PY{o}{=} \PY{n}{mpimg}\PY{o}{.}\PY{n}{imread}\PY{p}{(}\PY{l+s+s1}{\PYZsq{}}\PY{l+s+s1}{test\PYZus{}images/test1.jpg}\PY{l+s+s1}{\PYZsq{}}\PY{p}{)}
        \PY{n+nb}{print}\PY{p}{(}\PY{n}{corners\PYZus{}unwarp}\PY{p}{(}\PY{n}{img}\PY{p}{,} \PY{n}{nx}\PY{p}{,} \PY{n}{ny}\PY{p}{,} \PY{n}{mtx}\PY{p}{,} \PY{n}{dist}\PY{p}{)}\PY{p}{)}
\end{Verbatim}


    \begin{Verbatim}[commandchars=\\\{\}]

        ---------------------------------------------------------------------------

        UnboundLocalError                         Traceback (most recent call last)

        <ipython-input-6-c143702abac3> in <module>
         35 ny = 6
         36 img = mpimg.imread('test\_images/test1.jpg')
    ---> 37 print(corners\_unwarp(img, nx, ny, mtx, dist))
    

        <ipython-input-6-c143702abac3> in corners\_unwarp(img, nx, ny, mtx, dist)
         29 
         30     \# Return the resulting image and matrix
    ---> 31     return warped, M
         32 
         33 \# Testing the function on the sample image
    

        UnboundLocalError: local variable 'warped' referenced before assignment

    \end{Verbatim}

    \begin{Verbatim}[commandchars=\\\{\}]
{\color{incolor}In [{\color{incolor}11}]:}     \PY{n}{nx} \PY{o}{=} \PY{l+m+mi}{9}
             \PY{n}{ny} \PY{o}{=} \PY{l+m+mi}{6}
             \PY{n}{img} \PY{o}{=} \PY{n}{mpimg}\PY{o}{.}\PY{n}{imread}\PY{p}{(}\PY{l+s+s1}{\PYZsq{}}\PY{l+s+s1}{test\PYZus{}images/test1.jpg}\PY{l+s+s1}{\PYZsq{}}\PY{p}{)}
         
         
             \PY{n}{undist} \PY{o}{=} \PY{n}{cv2}\PY{o}{.}\PY{n}{undistort}\PY{p}{(}\PY{n}{img}\PY{p}{,} \PY{n}{mtx}\PY{p}{,} \PY{n}{dist}\PY{p}{,} \PY{k+kc}{None}\PY{p}{,} \PY{n}{mtx}\PY{p}{)}
                 \PY{c+c1}{\PYZsh{} Converting to grayscale}
             \PY{n}{gray} \PY{o}{=} \PY{n}{cv2}\PY{o}{.}\PY{n}{cvtColor}\PY{p}{(}\PY{n}{undist}\PY{p}{,} \PY{n}{cv2}\PY{o}{.}\PY{n}{COLOR\PYZus{}BGR2GRAY}\PY{p}{)}
                 \PY{c+c1}{\PYZsh{} Identifying corners (number of corners is predefined in the function call)}
             \PY{n}{ret}\PY{p}{,} \PY{n}{corners} \PY{o}{=} \PY{n}{cv2}\PY{o}{.}\PY{n}{findChessboardCorners}\PY{p}{(}\PY{n}{gray}\PY{p}{,} \PY{p}{(}\PY{n}{nx}\PY{p}{,} \PY{n}{ny}\PY{p}{)}\PY{p}{,} \PY{k+kc}{None}\PY{p}{)}
         
             \PY{k}{if} \PY{n}{ret} \PY{o}{==} \PY{k+kc}{True}\PY{p}{:}
                 \PY{n}{cv2}\PY{o}{.}\PY{n}{drawChessboardCorners}\PY{p}{(}\PY{n}{undist}\PY{p}{,} \PY{p}{(}\PY{n}{nx}\PY{p}{,} \PY{n}{ny}\PY{p}{)}\PY{p}{,} \PY{n}{corners}\PY{p}{,} \PY{n}{ret}\PY{p}{)}
                 \PY{c+c1}{\PYZsh{} Arbitrary offset for cropping to preserve the aspect ratio}
                 \PY{n}{offset} \PY{o}{=} \PY{l+m+mi}{100} 
                 \PY{n}{img\PYZus{}size} \PY{o}{=} \PY{p}{(}\PY{n}{gray}\PY{o}{.}\PY{n}{shape}\PY{p}{[}\PY{l+m+mi}{1}\PY{p}{]}\PY{p}{,} \PY{n}{gray}\PY{o}{.}\PY{n}{shape}\PY{p}{[}\PY{l+m+mi}{0}\PY{p}{]}\PY{p}{)}
         
                 \PY{c+c1}{\PYZsh{} Identifying the fringe corners}
                 \PY{n}{src} \PY{o}{=} \PY{n}{np}\PY{o}{.}\PY{n}{float32}\PY{p}{(}\PY{p}{[}\PY{n}{corners}\PY{p}{[}\PY{l+m+mi}{0}\PY{p}{]}\PY{p}{,} \PY{n}{corners}\PY{p}{[}\PY{n}{nx}\PY{o}{\PYZhy{}}\PY{l+m+mi}{1}\PY{p}{]}\PY{p}{,} \PY{n}{corners}\PY{p}{[}\PY{o}{\PYZhy{}}\PY{l+m+mi}{1}\PY{p}{]}\PY{p}{,} \PY{n}{corners}\PY{p}{[}\PY{o}{\PYZhy{}}\PY{n}{nx}\PY{p}{]}\PY{p}{]}\PY{p}{)}
                 \PY{c+c1}{\PYZsh{} Arbitrary four points based on offset crop}
                 \PY{n}{dst} \PY{o}{=} \PY{n}{np}\PY{o}{.}\PY{n}{float32}\PY{p}{(}\PY{p}{[}\PY{p}{[}\PY{n}{offset}\PY{p}{,} \PY{n}{offset}\PY{p}{]}\PY{p}{,} \PY{p}{[}\PY{n}{img\PYZus{}size}\PY{p}{[}\PY{l+m+mi}{0}\PY{p}{]}\PY{o}{\PYZhy{}}\PY{n}{offset}\PY{p}{,} \PY{n}{offset}\PY{p}{]}\PY{p}{,} 
                                              \PY{p}{[}\PY{n}{img\PYZus{}size}\PY{p}{[}\PY{l+m+mi}{0}\PY{p}{]}\PY{o}{\PYZhy{}}\PY{n}{offset}\PY{p}{,} \PY{n}{img\PYZus{}size}\PY{p}{[}\PY{l+m+mi}{1}\PY{p}{]}\PY{o}{\PYZhy{}}\PY{n}{offset}\PY{p}{]}\PY{p}{,} 
                                              \PY{p}{[}\PY{n}{offset}\PY{p}{,} \PY{n}{img\PYZus{}size}\PY{p}{[}\PY{l+m+mi}{1}\PY{p}{]}\PY{o}{\PYZhy{}}\PY{n}{offset}\PY{p}{]}\PY{p}{]}\PY{p}{)}
                 \PY{c+c1}{\PYZsh{} Extracting perspective transform matrix}
                 \PY{n}{M} \PY{o}{=} \PY{n}{cv2}\PY{o}{.}\PY{n}{getPerspectiveTransform}\PY{p}{(}\PY{n}{src}\PY{p}{,} \PY{n}{dst}\PY{p}{)}
                 \PY{c+c1}{\PYZsh{} Warping the image}
                 \PY{n}{warped} \PY{o}{=} \PY{n}{cv2}\PY{o}{.}\PY{n}{warpPerspective}\PY{p}{(}\PY{n}{undist}\PY{p}{,} \PY{n}{M}\PY{p}{,} \PY{n}{img\PYZus{}size}\PY{p}{)}
             \PY{n+nb}{print}\PY{p}{(}\PY{n}{ret}\PY{p}{)}
             \PY{n+nb}{print}\PY{p}{(}\PY{n}{warped}\PY{p}{)}
\end{Verbatim}


    \begin{Verbatim}[commandchars=\\\{\}]
False

    \end{Verbatim}

    \begin{Verbatim}[commandchars=\\\{\}]

        ---------------------------------------------------------------------------

        NameError                                 Traceback (most recent call last)

        <ipython-input-11-de473282a2b5> in <module>
         27     warped = cv2.warpPerspective(undist, M, img\_size)
         28 print(ret)
    ---> 29 print(warped)
    

        NameError: name 'warped' is not defined

    \end{Verbatim}

    \begin{Verbatim}[commandchars=\\\{\}]
{\color{incolor}In [{\color{incolor}32}]:} \PY{c+c1}{\PYZsh{} Testing calibration on the sample image}
         \PY{n+nb}{print}\PY{p}{(}\PY{n}{ret}\PY{p}{)}
         \PY{n+nb}{print}\PY{p}{(}\PY{n}{mtx}\PY{p}{)}
         \PY{n}{img} \PY{o}{=} \PY{n}{mpimg}\PY{o}{.}\PY{n}{imread}\PY{p}{(}\PY{l+s+s1}{\PYZsq{}}\PY{l+s+s1}{test\PYZus{}images/test1.jpg}\PY{l+s+s1}{\PYZsq{}}\PY{p}{)}
         \PY{k}{if} \PY{n}{ret} \PY{o}{==} \PY{k+kc}{True}\PY{p}{:}
             \PY{n}{dst} \PY{o}{=} \PY{n}{cv2}\PY{o}{.}\PY{n}{undistort}\PY{p}{(}\PY{n}{img}\PY{p}{,} \PY{n}{mtx}\PY{p}{,} \PY{n}{dist}\PY{p}{,} \PY{k+kc}{None}\PY{p}{,} \PY{n}{mtx}\PY{p}{)}
         \PY{n}{plt}\PY{o}{.}\PY{n}{imshow}\PY{p}{(}\PY{n}{img}\PY{p}{)}
         \PY{n}{plt}\PY{o}{.}\PY{n}{show}\PY{p}{(}\PY{p}{)}
\end{Verbatim}


    \begin{Verbatim}[commandchars=\\\{\}]
1.029815337105897
[[  1.15777930e+03   0.00000000e+00   6.67111054e+02]
 [  0.00000000e+00   1.15282291e+03   3.86128937e+02]
 [  0.00000000e+00   0.00000000e+00   1.00000000e+00]]

    \end{Verbatim}

    \begin{Verbatim}[commandchars=\\\{\}]

        ---------------------------------------------------------------------------

        TypeError                                 Traceback (most recent call last)

        <ipython-input-32-12b803556e20> in <module>
          6 if ret == True:
          7     dst = cv2.undistort(img, mtx, dist, None, mtx)
    ----> 8 plt.imshow(img)
          9 plt.show()
    

        \textasciitilde{}\textbackslash{}AppData\textbackslash{}Local\textbackslash{}Continuum\textbackslash{}miniconda3\textbackslash{}envs\textbackslash{}carnd-term1\textbackslash{}lib\textbackslash{}site-packages\textbackslash{}matplotlib\textbackslash{}pyplot.py in imshow(X, cmap, norm, aspect, interpolation, alpha, vmin, vmax, origin, extent, shape, filternorm, filterrad, imlim, resample, url, hold, data, **kwargs)
       3140            filternorm=1, filterrad=4.0, imlim=None, resample=None, url=None,
       3141            hold=None, data=None, **kwargs):
    -> 3142     ax = gca()
       3143     \# Deprecated: allow callers to override the hold state
       3144     \# by passing hold=True|False
    

        \textasciitilde{}\textbackslash{}AppData\textbackslash{}Local\textbackslash{}Continuum\textbackslash{}miniconda3\textbackslash{}envs\textbackslash{}carnd-term1\textbackslash{}lib\textbackslash{}site-packages\textbackslash{}matplotlib\textbackslash{}pyplot.py in gca(**kwargs)
        948     matplotlib.figure.Figure.gca : The figure's gca method.
        949     """
    --> 950     return gcf().gca(**kwargs)
        951 
        952 \# More ways of creating axes:
    

        \textasciitilde{}\textbackslash{}AppData\textbackslash{}Local\textbackslash{}Continuum\textbackslash{}miniconda3\textbackslash{}envs\textbackslash{}carnd-term1\textbackslash{}lib\textbackslash{}site-packages\textbackslash{}matplotlib\textbackslash{}pyplot.py in gcf()
        584         return figManager.canvas.figure
        585     else:
    --> 586         return figure()
        587 
        588 
    

        \textasciitilde{}\textbackslash{}AppData\textbackslash{}Local\textbackslash{}Continuum\textbackslash{}miniconda3\textbackslash{}envs\textbackslash{}carnd-term1\textbackslash{}lib\textbackslash{}site-packages\textbackslash{}matplotlib\textbackslash{}pyplot.py in figure(num, figsize, dpi, facecolor, edgecolor, frameon, FigureClass, **kwargs)
        533                                         frameon=frameon,
        534                                         FigureClass=FigureClass,
    --> 535                                         **kwargs)
        536 
        537         if figLabel:
    

        \textasciitilde{}\textbackslash{}AppData\textbackslash{}Local\textbackslash{}Continuum\textbackslash{}miniconda3\textbackslash{}envs\textbackslash{}carnd-term1\textbackslash{}lib\textbackslash{}site-packages\textbackslash{}matplotlib\textbackslash{}backends\textbackslash{}backend\_qt5agg.py in new\_figure\_manager(num, *args, **kwargs)
         42     FigureClass = kwargs.pop('FigureClass', Figure)
         43     thisFig = FigureClass(*args, **kwargs)
    ---> 44     return new\_figure\_manager\_given\_figure(num, thisFig)
         45 
         46 
    

        \textasciitilde{}\textbackslash{}AppData\textbackslash{}Local\textbackslash{}Continuum\textbackslash{}miniconda3\textbackslash{}envs\textbackslash{}carnd-term1\textbackslash{}lib\textbackslash{}site-packages\textbackslash{}matplotlib\textbackslash{}backends\textbackslash{}backend\_qt5agg.py in new\_figure\_manager\_given\_figure(num, figure)
         49     Create a new figure manager instance for the given figure.
         50     """
    ---> 51     canvas = FigureCanvasQTAgg(figure)
         52     return FigureManagerQT(canvas, num)
         53 
    

        \textasciitilde{}\textbackslash{}AppData\textbackslash{}Local\textbackslash{}Continuum\textbackslash{}miniconda3\textbackslash{}envs\textbackslash{}carnd-term1\textbackslash{}lib\textbackslash{}site-packages\textbackslash{}matplotlib\textbackslash{}backends\textbackslash{}backend\_qt5agg.py in \_\_init\_\_(self, figure)
        240         if DEBUG:
        241             print('FigureCanvasQtAgg: ', figure)
    --> 242         super(FigureCanvasQTAgg, self).\_\_init\_\_(figure=figure)
        243         self.\_drawRect = None
        244         self.blitbox = []
    

        \textasciitilde{}\textbackslash{}AppData\textbackslash{}Local\textbackslash{}Continuum\textbackslash{}miniconda3\textbackslash{}envs\textbackslash{}carnd-term1\textbackslash{}lib\textbackslash{}site-packages\textbackslash{}matplotlib\textbackslash{}backends\textbackslash{}backend\_qt5agg.py in \_\_init\_\_(self, figure)
         64 
         65     def \_\_init\_\_(self, figure):
    ---> 66         super(FigureCanvasQTAggBase, self).\_\_init\_\_(figure=figure)
         67         self.\_agg\_draw\_pending = False
         68 
    

        \textasciitilde{}\textbackslash{}AppData\textbackslash{}Local\textbackslash{}Continuum\textbackslash{}miniconda3\textbackslash{}envs\textbackslash{}carnd-term1\textbackslash{}lib\textbackslash{}site-packages\textbackslash{}matplotlib\textbackslash{}backends\textbackslash{}backend\_qt5.py in \_\_init\_\_(self, figure)
        238         \# The need for this change is documented here
        239         \# http://pyqt.sourceforge.net/Docs/PyQt5/pyqt4\_differences.html\#cooperative-multi-inheritance
    --> 240         super(FigureCanvasQT, self).\_\_init\_\_(figure=figure)
        241         self.figure = figure
        242         self.setMouseTracking(True)
    

        TypeError: 'figure' is an unknown keyword argument

    \end{Verbatim}

    \begin{Verbatim}[commandchars=\\\{\}]
{\color{incolor}In [{\color{incolor}47}]:} \PY{c+c1}{\PYZsh{}cv2.destroyAllWindows()}
         \PY{c+c1}{\PYZsh{}reading in an image}
         \PY{n}{image} \PY{o}{=} \PY{n}{mpimg}\PY{o}{.}\PY{n}{imread}\PY{p}{(}\PY{l+s+s1}{\PYZsq{}}\PY{l+s+s1}{test\PYZus{}images/test1.jpg}\PY{l+s+s1}{\PYZsq{}}\PY{p}{)}
         \PY{c+c1}{\PYZsh{}printing out some stats and plotting}
         \PY{n+nb}{print}\PY{p}{(}\PY{l+s+s1}{\PYZsq{}}\PY{l+s+s1}{This image is:}\PY{l+s+s1}{\PYZsq{}}\PY{p}{,} \PY{n+nb}{type}\PY{p}{(}\PY{n}{image}\PY{p}{)}\PY{p}{,} \PY{l+s+s1}{\PYZsq{}}\PY{l+s+s1}{with dimensions:}\PY{l+s+s1}{\PYZsq{}}\PY{p}{,} \PY{n}{image}\PY{o}{.}\PY{n}{shape}\PY{p}{)}
         
         \PY{k+kn}{import} \PY{n+nn}{matplotlib}
         \PY{n}{matplotlib}\PY{o}{.}\PY{n}{use}\PY{p}{(}\PY{l+s+s1}{\PYZsq{}}\PY{l+s+s1}{Qt4Agg}\PY{l+s+s1}{\PYZsq{}}\PY{p}{)}
         \PY{k+kn}{from} \PY{n+nn}{matplotlib} \PY{k}{import} \PY{n}{pyplot} \PY{k}{as} \PY{n}{plt}
         
         \PY{c+c1}{\PYZsh{}plt.figure(figsize=(12,8))}
         \PY{c+c1}{\PYZsh{}plt.title(\PYZdq{}Score\PYZdq{})}
         \PY{c+c1}{\PYZsh{}plt.show()}
         
         \PY{n}{plt}\PY{o}{.}\PY{n}{imshow}\PY{p}{(}\PY{n}{image}\PY{p}{)}  \PY{c+c1}{\PYZsh{} if you wanted to show a single color channel image called \PYZsq{}gray\PYZsq{}, for example, call as plt.imshow(gray, cmap=\PYZsq{}gray\PYZsq{})}
\end{Verbatim}


    \begin{Verbatim}[commandchars=\\\{\}]
This image is: <class 'numpy.ndarray'> with dimensions: (720, 1280, 3)

    \end{Verbatim}

    \begin{Verbatim}[commandchars=\\\{\}]

        ---------------------------------------------------------------------------

        TypeError                                 Traceback (most recent call last)

        <ipython-input-47-aef0059559b6> in <module>
         13 \#plt.show()
         14 
    ---> 15 plt.imshow(image)  \# if you wanted to show a single color channel image called 'gray', for example, call as plt.imshow(gray, cmap='gray')
    

        \textasciitilde{}\textbackslash{}AppData\textbackslash{}Local\textbackslash{}Continuum\textbackslash{}miniconda3\textbackslash{}envs\textbackslash{}carnd-term1\textbackslash{}lib\textbackslash{}site-packages\textbackslash{}matplotlib\textbackslash{}pyplot.py in imshow(X, cmap, norm, aspect, interpolation, alpha, vmin, vmax, origin, extent, shape, filternorm, filterrad, imlim, resample, url, hold, data, **kwargs)
       3140            filternorm=1, filterrad=4.0, imlim=None, resample=None, url=None,
       3141            hold=None, data=None, **kwargs):
    -> 3142     ax = gca()
       3143     \# Deprecated: allow callers to override the hold state
       3144     \# by passing hold=True|False
    

        \textasciitilde{}\textbackslash{}AppData\textbackslash{}Local\textbackslash{}Continuum\textbackslash{}miniconda3\textbackslash{}envs\textbackslash{}carnd-term1\textbackslash{}lib\textbackslash{}site-packages\textbackslash{}matplotlib\textbackslash{}pyplot.py in gca(**kwargs)
        948     matplotlib.figure.Figure.gca : The figure's gca method.
        949     """
    --> 950     return gcf().gca(**kwargs)
        951 
        952 \# More ways of creating axes:
    

        \textasciitilde{}\textbackslash{}AppData\textbackslash{}Local\textbackslash{}Continuum\textbackslash{}miniconda3\textbackslash{}envs\textbackslash{}carnd-term1\textbackslash{}lib\textbackslash{}site-packages\textbackslash{}matplotlib\textbackslash{}pyplot.py in gcf()
        584         return figManager.canvas.figure
        585     else:
    --> 586         return figure()
        587 
        588 
    

        \textasciitilde{}\textbackslash{}AppData\textbackslash{}Local\textbackslash{}Continuum\textbackslash{}miniconda3\textbackslash{}envs\textbackslash{}carnd-term1\textbackslash{}lib\textbackslash{}site-packages\textbackslash{}matplotlib\textbackslash{}pyplot.py in figure(num, figsize, dpi, facecolor, edgecolor, frameon, FigureClass, **kwargs)
        533                                         frameon=frameon,
        534                                         FigureClass=FigureClass,
    --> 535                                         **kwargs)
        536 
        537         if figLabel:
    

        \textasciitilde{}\textbackslash{}AppData\textbackslash{}Local\textbackslash{}Continuum\textbackslash{}miniconda3\textbackslash{}envs\textbackslash{}carnd-term1\textbackslash{}lib\textbackslash{}site-packages\textbackslash{}matplotlib\textbackslash{}backends\textbackslash{}backend\_qt5agg.py in new\_figure\_manager(num, *args, **kwargs)
         42     FigureClass = kwargs.pop('FigureClass', Figure)
         43     thisFig = FigureClass(*args, **kwargs)
    ---> 44     return new\_figure\_manager\_given\_figure(num, thisFig)
         45 
         46 
    

        \textasciitilde{}\textbackslash{}AppData\textbackslash{}Local\textbackslash{}Continuum\textbackslash{}miniconda3\textbackslash{}envs\textbackslash{}carnd-term1\textbackslash{}lib\textbackslash{}site-packages\textbackslash{}matplotlib\textbackslash{}backends\textbackslash{}backend\_qt5agg.py in new\_figure\_manager\_given\_figure(num, figure)
         49     Create a new figure manager instance for the given figure.
         50     """
    ---> 51     canvas = FigureCanvasQTAgg(figure)
         52     return FigureManagerQT(canvas, num)
         53 
    

        \textasciitilde{}\textbackslash{}AppData\textbackslash{}Local\textbackslash{}Continuum\textbackslash{}miniconda3\textbackslash{}envs\textbackslash{}carnd-term1\textbackslash{}lib\textbackslash{}site-packages\textbackslash{}matplotlib\textbackslash{}backends\textbackslash{}backend\_qt5agg.py in \_\_init\_\_(self, figure)
        240         if DEBUG:
        241             print('FigureCanvasQtAgg: ', figure)
    --> 242         super(FigureCanvasQTAgg, self).\_\_init\_\_(figure=figure)
        243         self.\_drawRect = None
        244         self.blitbox = []
    

        \textasciitilde{}\textbackslash{}AppData\textbackslash{}Local\textbackslash{}Continuum\textbackslash{}miniconda3\textbackslash{}envs\textbackslash{}carnd-term1\textbackslash{}lib\textbackslash{}site-packages\textbackslash{}matplotlib\textbackslash{}backends\textbackslash{}backend\_qt5agg.py in \_\_init\_\_(self, figure)
         64 
         65     def \_\_init\_\_(self, figure):
    ---> 66         super(FigureCanvasQTAggBase, self).\_\_init\_\_(figure=figure)
         67         self.\_agg\_draw\_pending = False
         68 
    

        \textasciitilde{}\textbackslash{}AppData\textbackslash{}Local\textbackslash{}Continuum\textbackslash{}miniconda3\textbackslash{}envs\textbackslash{}carnd-term1\textbackslash{}lib\textbackslash{}site-packages\textbackslash{}matplotlib\textbackslash{}backends\textbackslash{}backend\_qt5.py in \_\_init\_\_(self, figure)
        238         \# The need for this change is documented here
        239         \# http://pyqt.sourceforge.net/Docs/PyQt5/pyqt4\_differences.html\#cooperative-multi-inheritance
    --> 240         super(FigureCanvasQT, self).\_\_init\_\_(figure=figure)
        241         self.figure = figure
        242         self.setMouseTracking(True)
    

        TypeError: 'figure' is an unknown keyword argument

    \end{Verbatim}

    \hypertarget{and-so-on-and-so-forth}{%
\subsection{And so on and so
forth\ldots{}}\label{and-so-on-and-so-forth}}

    \begin{Verbatim}[commandchars=\\\{\}]
{\color{incolor}In [{\color{incolor} }]:} \PY{k}{def} \PY{n+nf}{warper}\PY{p}{(}\PY{n}{img}\PY{p}{,} \PY{n}{src}\PY{p}{,} \PY{n}{dst}\PY{p}{)}\PY{p}{:}
        
            \PY{c+c1}{\PYZsh{} Compute and apply perpective transform}
            \PY{n}{img\PYZus{}size} \PY{o}{=} \PY{p}{(}\PY{n}{img}\PY{o}{.}\PY{n}{shape}\PY{p}{[}\PY{l+m+mi}{1}\PY{p}{]}\PY{p}{,} \PY{n}{img}\PY{o}{.}\PY{n}{shape}\PY{p}{[}\PY{l+m+mi}{0}\PY{p}{]}\PY{p}{)}
            \PY{n}{M} \PY{o}{=} \PY{n}{cv2}\PY{o}{.}\PY{n}{getPerspectiveTransform}\PY{p}{(}\PY{n}{src}\PY{p}{,} \PY{n}{dst}\PY{p}{)}
            \PY{n}{warped} \PY{o}{=} \PY{n}{cv2}\PY{o}{.}\PY{n}{warpPerspective}\PY{p}{(}\PY{n}{img}\PY{p}{,} \PY{n}{M}\PY{p}{,} \PY{n}{img\PYZus{}size}\PY{p}{,} \PY{n}{flags}\PY{o}{=}\PY{n}{cv2}\PY{o}{.}\PY{n}{INTER\PYZus{}NEAREST}\PY{p}{)}  \PY{c+c1}{\PYZsh{} keep same size as input image}
        
            \PY{k}{return} \PY{n}{warped}
\end{Verbatim}



    % Add a bibliography block to the postdoc
    
    
    
    \end{document}
